\documentclass{article}

\usepackage{booktabs}
\usepackage{tabularx}
\usepackage{hyperref}

\hypersetup{
    colorlinks=true,       % false: boxed links; true: colored links
    linkcolor=red,          % color of internal links (change box color with linkbordercolor)
    citecolor=green,        % color of links to bibliography
    filecolor=magenta,      % color of file links
    urlcolor=cyan           % color of external links
}

\title{Hazard Analysis\\\progname}

\author{\authname}

\date{}

%% Comments

\usepackage{color}

\newif\ifcomments\commentstrue %displays comments
%\newif\ifcomments\commentsfalse %so that comments do not display

\ifcomments
\newcommand{\authornote}[3]{\textcolor{#1}{[#3 ---#2]}}
\newcommand{\todo}[1]{\textcolor{red}{[TODO: #1]}}
\else
\newcommand{\authornote}[3]{}
\newcommand{\todo}[1]{}
\fi

\newcommand{\wss}[1]{\authornote{magenta}{SS}{#1}} 
\newcommand{\plt}[1]{\authornote{cyan}{TPLT}{#1}} %For explanation of the template
\newcommand{\an}[1]{\authornote{cyan}{Author}{#1}}

%% Common Parts

\newcommand{\progname}{Software Engineering} % PUT YOUR PROGRAM NAME HERE
\newcommand{\authname}{Team 21, Visionaries
\\ Ahmed Sahi
\\ Angela Zeng
\\ Ann Shi
\\ Manan Sharma
\\ Stanley Chen} % AUTHOR NAMES                  

\usepackage{hyperref}
    \hypersetup{colorlinks=true, linkcolor=blue, citecolor=blue, filecolor=blue,
                urlcolor=blue, unicode=false}
    \urlstyle{same}
                                


\begin{document}

\maketitle
\thispagestyle{empty}

~\newpage

\pagenumbering{roman}

\begin{table}[hp]
\caption{Revision History} \label{TblRevisionHistory}
\begin{tabularx}{\textwidth}{llX}
\toprule
\textbf{Date} & \textbf{Developer(s)} & \textbf{Change}\\
\midrule
Date1 & Name(s) & Description of changes\\
Date2 & Name(s) & Description of changes\\
... & ... & ...\\
\bottomrule
\end{tabularx}
\end{table}

~\newpage

\tableofcontents

~\newpage

\pagenumbering{arabic}

\wss{You are free to modify this template.}

\section{Introduction}

The following document contains an overview of the hazards highlighted in the Large-Group Eye-Tracking Capstone Project. For the purposes of this document, a hazard is (based on the work of Nancy Leveson) defined as any aspect or property of this project which causes harm, damage or loss in the environment the system inhabits. This document identifies key hazards involved, and uses the Failure Modes and Effects Analysis (FMEA) method to analyze them and their respective impacts on the system. 

\section{Scope and Purpose of Hazard Analysis}

The purpose of this Hazard Analysis is identifying system properties which may cause harm to stakeholders. In order to narrow the scope of this assessment, the following potential losses have been highlighted:
\begin{itemize}
    \item Privacy: unauthorized access, re-identification, misuse of gaze data
    \item Participant discomfort
    \item Data inaccuracy: invalid findings and conclusions
    \item Loss of stakeholder value: instructors receiving inaccurate or unusable real-time data
    \item Disrupting live classroom activities
    \end{itemize}

\section{System Boundaries and Components}

The proposed system is a learning platform that integrates large-group eye tracking with classroom activities, allowing instructors to view aggregated gaze information in real time and after class. To perform a meaningful hazard analysis, the system is divided into the following components:\\
\textbf{Data Collection (Eye Tracking)}\\
The core of this project is the data retrieved using the Eye Tracking hardware, specifically in a classroom setting. This first component includes both the tracking devices, as well as associated software used to record and store the data. A combination of the visual stimuli in the classroom (i.e. slideshows on a projected screen) and the eye movements of multiple students are used to return raw gaze data.\\
\textbf{Supplementary Data (Student Survey)}\\
A self-report style survey will be used to gather supplementary data from participants, such as demographics, learning-related disabilities and behavioural traits. Student-entered responses to pre- and post-lecture questionnaires will be converted to a structured supplementary dataset, linked to the central eye-tracking data.\\
\textbf{Data Analysis (Mapping Gaze Data)}\\
In this component, the raw gaze data is processed and mapped to the visual stimulus to determine attention patterns, creating quantitative results which can be visualized in real-time. When assessing this alongside the supplementary data, more specific correlations can be determined.\\
\textbf{Dashboards (Instructor Visualization Interfaces)}\\
This component can be broken up into two further components: visual outputs to instructors in real time (during class), and afterward (for review). The processed and analysed outputs are translated to interactive and easily interpretable visualizations.


\section{Critical Assumptions}

This analysis assumes that the eye-tracking hardware functions reliably and that hardware-level failures are out of scope. It also assumes that the network and server infrastructure are stable and secure, with standard IT reliability already in place.

\section{Failure Mode and Effect Analysis}

\wss{Include your FMEA table here. This is the most important part of this document.}
\wss{The safety requirements in the table do not have to have the prefix SR.
The most important thing is to show traceability to your SRS. You might trace to
requirements you have already written, or you might need to add new
requirements.}
\wss{If no safety requirement can be devised, other mitigation strategies can be
entered in the table, including strategies involving providing additional
documentation, and/or test cases.}

\section{Safety and Security Requirements}

\wss{Newly discovered requirements.  These should also be added to the SRS.  (A
rationale design process how and why to fake it.)}

\section{Roadmap}

\wss{Which safety requirements will be implemented as part of the capstone timeline?
Which requirements will be implemented in the future?}

\newpage{}

\section*{Appendix --- Reflection}

\wss{Not required for CAS 741}

The purpose of reflection questions is to give you a chance to assess your own
learning and that of your group as a whole, and to find ways to improve in the
future. Reflection is an important part of the learning process.  Reflection is
also an essential component of a successful software development process.  

Reflections are most interesting and useful when they're honest, even if the
stories they tell are imperfect. You will be marked based on your depth of
thought and analysis, and not based on the content of the reflections
themselves. Thus, for full marks we encourage you to answer openly and honestly
and to avoid simply writing ``what you think the evaluator wants to hear.''

Please answer the following questions.  Some questions can be answered on the
team level, but where appropriate, each team member should write their own
response:


\begin{enumerate}
    \item What went well while writing this deliverable? 
    \item What pain points did you experience during this deliverable, and how
    did you resolve them?
    \item Which of your listed risks had your team thought of before this
    deliverable, and which did you think of while doing this deliverable? For
    the latter ones (ones you thought of while doing the Hazard Analysis), how
    did they come about?
    \item Other than the risk of physical harm (some projects may not have any
    appreciable risks of this form), list at least 2 other types of risk in
    software products. Why are they important to consider?
\end{enumerate}

\end{document}

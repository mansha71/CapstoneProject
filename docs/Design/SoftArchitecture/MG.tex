\documentclass[12pt, titlepage]{article}

\usepackage{fullpage}
\usepackage[round]{natbib}
\usepackage{multirow}
\usepackage{booktabs}
\usepackage{tabularx}
\usepackage{graphicx}
\usepackage{float}
\usepackage{hyperref}
\hypersetup{
    colorlinks,
    citecolor=blue,
    filecolor=black,
    linkcolor=red,
    urlcolor=blue
}

\input{../../Comments}
%% Common Parts

\newcommand{\progname}{Software Engineering} % PUT YOUR PROGRAM NAME HERE
\newcommand{\authname}{Team 21, Visionaries
\\ Angela Zeng
\\ Ann Shi
\\ Ibrahim Sahi
\\ Manan Sharma
\\ Stanley Chen} % AUTHOR NAMES                  

\usepackage{hyperref}
    \hypersetup{colorlinks=true, linkcolor=blue, citecolor=blue, filecolor=blue,
                urlcolor=blue, unicode=false}
    \urlstyle{same}
                                


\newcounter{acnum}
\newcommand{\actheacnum}{AC\theacnum}
\newcommand{\acref}[1]{AC\ref{#1}}

\newcounter{ucnum}
\newcommand{\uctheucnum}{UC\theucnum}
\newcommand{\uref}[1]{UC\ref{#1}}

\newcounter{mnum}
\newcommand{\mthemnum}{M\themnum}
\newcommand{\mref}[1]{M\ref{#1}}

\begin{document}

\title{Module Guide for \progname{}} 
\author{\authname}
\date{\today}

\maketitle

\pagenumbering{roman}

\section{Revision History}

\begin{tabularx}{\textwidth}{p{3cm}p{2cm}X}
\toprule {\bf Date} & {\bf Version} & {\bf Notes}\\
\midrule
Date 1 & 1.0 & Notes\\
Date 2 & 1.1 & Notes\\
\bottomrule
\end{tabularx}

\newpage

\section{Reference Material}

This section records information for easy reference.

\subsection{Abbreviations and Acronyms}

\renewcommand{\arraystretch}{1.2}
\begin{tabular}{l l} 
  \toprule		
  \textbf{symbol} & \textbf{description}\\
  \midrule 
  AC & Anticipated Change\\
  DAG & Directed Acyclic Graph \\
  M & Module \\
  MG & Module Guide \\
  OS & Operating System \\
  R & Requirement\\
  SC & Scientific Computing \\
  SRS & Software Requirements Specification\\
  \progname & Explanation of program name\\
  UC & Unlikely Change \\
  \wss{etc.} & \wss{...}\\
  \bottomrule
\end{tabular}\\

\newpage

\tableofcontents

\listoftables

\listoffigures

\newpage

\pagenumbering{arabic}

\section{Introduction}

Decomposing a system into modules is a commonly accepted approach to developing
software.  A module is a work assignment for a programmer or programming
team~\citep{ParnasEtAl1984}.  We advocate a decomposition
based on the principle of information hiding~\citep{Parnas1972a}.  This
principle supports design for change, because the ``secrets'' that each module
hides represent likely future changes.  Design for change is valuable in SC,
where modifications are frequent, especially during initial development as the
solution space is explored.  

Our design follows the rules layed out by \citet{ParnasEtAl1984}, as follows:
\begin{itemize}
\item System details that are likely to change independently should be the
  secrets of separate modules.
\item Each data structure is implemented in only one module.
\item Any other program that requires information stored in a module's data
  structures must obtain it by calling access programs belonging to that module.
\end{itemize}

After completing the first stage of the design, the Software Requirements
Specification (SRS), the Module Guide (MG) is developed~\citep{ParnasEtAl1984}. The MG
specifies the modular structure of the system and is intended to allow both
designers and maintainers to easily identify the parts of the software.  The
potential readers of this document are as follows:

\begin{itemize}
\item New project members: This document can be a guide for a new project member
  to easily understand the overall structure and quickly find the
  relevant modules they are searching for.
\item Maintainers: The hierarchical structure of the module guide improves the
  maintainers' understanding when they need to make changes to the system. It is
  important for a maintainer to update the relevant sections of the document
  after changes have been made.
\item Designers: Once the module guide has been written, it can be used to
  check for consistency, feasibility, and flexibility. Designers can verify the
  system in various ways, such as consistency among modules, feasibility of the
  decomposition, and flexibility of the design.
\end{itemize}

The rest of the document is organized as follows. Section
\ref{SecChange} lists the anticipated and unlikely changes of the software
requirements. Section \ref{SecMH} summarizes the module decomposition that
was constructed according to the likely changes. Section \ref{SecConnection}
specifies the connections between the software requirements and the
modules. Section \ref{SecMD} gives a detailed description of the
modules. Section \ref{SecTM} includes two traceability matrices. One checks
the completeness of the design against the requirements provided in the SRS. The
other shows the relation between anticipated changes and the modules. Section
\ref{SecUse} describes the use relation between modules.

\section{Anticipated and Unlikely Changes} \label{SecChange}

This section lists possible changes to the system. According to the likeliness
of the change, the possible changes are classified into two
categories. Anticipated changes are listed in Section \ref{SecAchange}, and
unlikely changes are listed in Section \ref{SecUchange}.

\subsection{Anticipated Changes} \label{SecAchange}

Anticipated changes are the source of the information that is to be hidden
inside the modules. Ideally, changing one of the anticipated changes will only
require changing the one module that hides the associated decision. The approach
adapted here is called design for
change.

\begin{description}
\item[\refstepcounter{acnum} \actheacnum \label{acHardware}:] The specific
  hardware on which the software is running.
\item[\refstepcounter{acnum} \actheacnum \label{acInput}:] The format of the
  initial input data.
\item ...
\end{description}

\wss{Anticipated changes relate to changes that would be made in requirements,
design or implementation choices.  They are not related to changes that are made
at run-time, like the values of parameters.}

\subsection{Unlikely Changes} \label{SecUchange}

The module design should be as general as possible. However, a general system is
more complex. Sometimes this complexity is not necessary. Fixing some design
decisions at the system architecture stage can simplify the software design. If
these decision should later need to be changed, then many parts of the design
will potentially need to be modified. Hence, it is not intended that these
decisions will be changed.

\begin{description}
\item[\refstepcounter{ucnum} \uctheucnum \label{ucIO}:] Input/Output devices
  (Input: File and/or Keyboard, Output: File, Memory, and/or Screen).
\item ...
\end{description}

\section{Module Hierarchy} \label{SecMH}

This section provides an overview of the module design. Modules are summarized
in a hierarchy decomposed by secrets in Table \ref{TblMH}. The modules listed
below, which are leaves in the hierarchy tree, are the modules that will
actually be implemented.

\begin{description}
\item [\refstepcounter{mnum} \mthemnum \label{mHH}:] Hardware-Hiding Module
\item ...
\end{description}


\begin{table}[h!]
\centering
\begin{tabular}{p{0.3\textwidth} p{0.6\textwidth}}
\toprule
\textbf{Level 1} & \textbf{Level 2}\\
\midrule

{Hardware-Hiding Module} & ~ \\
\midrule

\multirow{7}{0.3\textwidth}{Behaviour-Hiding Module} & ?\\
& ?\\
& ?\\
& ?\\
& ?\\
& ?\\
& ?\\ 
& ?\\
\midrule

\multirow{3}{0.3\textwidth}{Software Decision Module} & {?}\\
& ?\\
& ?\\
\bottomrule

\end{tabular}
\caption{Module Hierarchy}
\label{TblMH}
\end{table}

\section{Connection Between Requirements and Design} \label{SecConnection}

The design of the system is intended to satisfy the requirements developed in
the SRS. In this stage, the system is decomposed into modules. The connection
between requirements and modules is listed in Table~\ref{TblRT}.

\wss{The intention of this section is to document decisions that are made
  ``between'' the requirements and the design.  To satisfy some requirements,
  design decisions need to be made.  Rather than make these decisions implicit,
  they are explicitly recorded here.  For instance, if a program has security
  requirements, a specific design decision may be made to satisfy those
  requirements with a password.}

\section{Module Decomposition} \label{SecMD}

Modules are decomposed according to the principle of ``information hiding''
proposed by \citet{ParnasEtAl1984}. The \emph{Secrets} field in a module
decomposition is a brief statement of the design decision hidden by the
module. The \emph{Services} field specifies \emph{what} the module will do
without documenting \emph{how} to do it. For each module, a suggestion for the
implementing software is given under the \emph{Implemented By} title. If the
entry is \emph{OS}, this means that the module is provided by the operating
system or by standard programming language libraries.  \emph{\progname{}} means the
module will be implemented by the \progname{} software.

Only the leaf modules in the hierarchy have to be implemented. If a dash
(\emph{--}) is shown, this means that the module is not a leaf and will not have
to be implemented.

\subsection{Hardware Hiding Modules (\mref{mHH})}

\begin{description}
\item[Secrets:]The data structure and algorithm used to implement the virtual
  hardware.
\item[Services:]Serves as a virtual hardware used by the rest of the
  system. This module provides the interface between the hardware and the
  software. So, the system can use it to display outputs or to accept inputs.
\item[Implemented By:] OS
\end{description}

\subsection{Behaviour-Hiding Module}

\begin{description}
\item[Secrets:]The contents of the required behaviours.
\item[Services:]Includes programs that provide externally visible behaviour of
  the system as specified in the software requirements specification (SRS)
  documents. This module serves as a communication layer between the
  hardware-hiding module and the software decision module. The programs in this
  module will need to change if there are changes in the SRS.
\item[Implemented By:] --
\end{description}

\subsubsection{Input Format Module (\mref{mInput})}

\begin{description}
\item[Secrets:]The format and structure of the input data.
\item[Services:]Converts the input data into the data structure used by the
  input parameters module.
\item[Implemented By:] [Your Program Name Here]
\item[Type of Module:] [Record, Library, Abstract Object, or Abstract Data Type]
  [Information to include for leaf modules in the decomposition by secrets tree.]
\end{description}

\subsubsection{Etc.}


\subsection{Software Decision Module}

\begin{description}
\item[Secrets:] The design decision based on mathematical theorems, physical
  facts, or programming considerations. The secrets of this module are
  \emph{not} described in the SRS.
\item[Services:] Includes data structure and algorithms used in the system that
  do not provide direct interaction with the user. 
  % Changes in these modules are more likely to be motivated by a desire to
  % improve performance than by externally imposed changes.
\item[Implemented By:] --
\end{description}

\subsubsection{Etc.}

\section{Traceability Matrix} \label{SecTM}

This section shows two traceability matrices: between the modules and the
requirements and between the modules and the anticipated changes.

% the table should use mref, the requirements should be named, use something
% like fref
\begin{table}[H]
\centering
\begin{tabular}{p{0.2\textwidth} p{0.6\textwidth}}
\toprule
\textbf{Req.} & \textbf{Modules}\\
\midrule
R1 & \mref{mHH}, \mref{mInput}, \mref{mParams}, \mref{mControl}\\
R2 & \mref{mInput}, \mref{mParams}\\
R3 & \mref{mVerify}\\
R4 & \mref{mOutput}, \mref{mControl}\\
R5 & \mref{mOutput}, \mref{mODEs}, \mref{mControl}, \mref{mSeqDS}, \mref{mSolver}, \mref{mPlot}\\
R6 & \mref{mOutput}, \mref{mODEs}, \mref{mControl}, \mref{mSeqDS}, \mref{mSolver}, \mref{mPlot}\\
R7 & \mref{mOutput}, \mref{mEnergy}, \mref{mControl}, \mref{mSeqDS}, \mref{mPlot}\\
R8 & \mref{mOutput}, \mref{mEnergy}, \mref{mControl}, \mref{mSeqDS}, \mref{mPlot}\\
R9 & \mref{mVerifyOut}\\
R10 & \mref{mOutput}, \mref{mODEs}, \mref{mControl}\\
R11 & \mref{mOutput}, \mref{mODEs}, \mref{mEnergy}, \mref{mControl}\\
\bottomrule
\end{tabular}
\caption{Trace Between Requirements and Modules}
\label{TblRT}
\end{table}

\begin{table}[H]
\centering
\begin{tabular}{p{0.2\textwidth} p{0.6\textwidth}}
\toprule
\textbf{AC} & \textbf{Modules}\\
\midrule
\acref{acHardware} & \mref{mHH}\\
\acref{acInput} & \mref{mInput}\\
\acref{acParams} & \mref{mParams}\\
\acref{acVerify} & \mref{mVerify}\\
\acref{acOutput} & \mref{mOutput}\\
\acref{acVerifyOut} & \mref{mVerifyOut}\\
\acref{acODEs} & \mref{mODEs}\\
\acref{acEnergy} & \mref{mEnergy}\\
\acref{acControl} & \mref{mControl}\\
\acref{acSeqDS} & \mref{mSeqDS}\\
\acref{acSolver} & \mref{mSolver}\\
\acref{acPlot} & \mref{mPlot}\\
\bottomrule
\end{tabular}
\caption{Trace Between Anticipated Changes and Modules}
\label{TblACT}
\end{table}

\section{Use Hierarchy Between Modules} \label{SecUse}

In this section, the uses hierarchy between modules is
provided. \citet{Parnas1978} said of two programs A and B that A {\em uses} B if
correct execution of B may be necessary for A to complete the task described in
its specification. That is, A {\em uses} B if there exist situations in which
the correct functioning of A depends upon the availability of a correct
implementation of B.  Figure \ref{FigUH} illustrates the use relation between
the modules. It can be seen that the graph is a directed acyclic graph
(DAG). Each level of the hierarchy offers a testable and usable subset of the
system, and modules in the higher level of the hierarchy are essentially simpler
because they use modules from the lower levels.

At a high level, the use hierarchy for \progname{} is organized into layers:

\begin{itemize}
  \item \textbf{Level 0 – Hardware layer}: 
  Hardware-Hiding Module (M1) encapsulates the details of the Pupil Labs Neon
  glasses, central camera, and instructor workstation. No other module is used
  by M1.

  \item \textbf{Level 1 – Data acquisition}: 
  Data Ingestion Module (M2) uses M1 to collect raw gaze and video streams from
  the eye-tracking hardware and central camera.

  \item \textbf{Level 2 – Preparation and infrastructure}: 
  Data Preprocessing Module (M6) uses M2 to filter, normalize, and structure
  the captured data.  
  Privacy \& Infrastructure Module (M7) uses M1, M2, and M6 to apply
  anonymization, access control, and secure storage over the ingested and
  preprocessed data.

  \item \textbf{Level 3 – Analytics and streaming}: 
  Real-Time Streaming Module (M3) uses M2, M6, and M7 to serve live, privacy-
  compliant streams to the dashboard.  
  Engagement Analytics Module (M8) uses M6 to compute engagement metrics over
  preprocessed data.  
  Correlation \& Visual Analysis Module (M9) uses M6 and M8 to compute
  gaze-correlation and spatial visual analytics.

  \item \textbf{Level 4 – User-facing visualization}: 
  Dashboard Visualization Module (M4) uses M3, M7, M8, and M9 to display
  real-time and historical analytics while respecting privacy constraints.

  \item \textbf{Level 5 – Reporting and exports}: 
  Reporting Module (M5) uses M4, M8, and M9 to generate summaries and exported
  reports based on the visualizations and analytics.
\end{itemize}

The main uses relations can be summarized as:

\begin{itemize}
  \item M2 uses M1.
  \item M6 uses M2.
  \item M7 uses M1, M2, and M6.
  \item M3 uses M2, M6, and M7.
  \item M8 uses M6.
  \item M9 uses M6 and M8.
  \item M4 uses M3, M7, M8, and M9.
  \item M5 uses M4, M8, and M9.
\end{itemize}

\begin{figure}[H]
\centering
%\includegraphics[width=0.7\textwidth]{UsesHierarchy.png}
\caption{Use hierarchy among modules}
\label{FigUH}
\end{figure}

%\section*{References}

\section{User Interfaces}

This section provides conceptual sketches of the primary user interface screens
for \progname{}. These sketches represent the core interactions and visual layout
of the system. They are intended to illustrate how instructors will navigate
through the platform, monitor real-time classroom attention, and review analytics
after each session. The sketches are early design mockups and may evolve as the
implementation progresses.

\subsection{10.1 Student Overview}

\begin{figure}[H]
\centering
\includegraphics[width=0.55\textwidth]{figure/Student_Overview.png}
\caption{Student Overview Screen}
\label{fig:student-overview}
\end{figure}

This screen serves as the entry point to the platform. Instructors can begin a
new live session or review summaries of previously recorded sessions.

\subsection{10.2 Live Classroom Dashboard}

\begin{figure}[H]
\centering
\includegraphics[width=0.55\textwidth]{figure/Live-Classroom-Dashboard.png}
\caption{Live Classroom Dashboard}
\label{fig:live-dashboard}
\end{figure}

During an active session, instructors are presented with a real-time dashboard.
It displays the recording state, a live camera feed from the capture device, and
a heatmap preview showing the current distribution of gaze activity.

\subsection{10.3 Student Attention Panel}

\begin{figure}[H]
\centering
\includegraphics[width=0.55\textwidth]{figure/Student_Attention_Panel.png}
\caption{Student Attention Panel}
\label{fig:student-attention-panel}
\end{figure}

This panel provides a real-time overview of individual student attention levels.
Each student is listed with a visual bar indicator, allowing the instructor to
quickly identify who is highly engaged, who is disengaged, and how attention
varies across learners.

\subsection{10.4 Selected Student Detail View}

\begin{figure}[H]
\centering
\includegraphics[width=0.55\textwidth]{figure/Selected_Student.png}
\caption{Selected Student Detail View}
\label{fig:selected-student}
\end{figure}

When a specific student is selected, the instructor is shown a detailed breakdown
of that student’s gaze distribution across key classroom regions (e.g., board,
instructor, walls). A small trend graph summarizes how their engagement changes
over time.

\subsection{10.5 Region-of-Interest Analytics}

\begin{figure}[H]
\centering
\includegraphics[width=0.55\textwidth]{figure/Region_of_Interest_Analytics.png}
\caption{Region-of-Interest (ROI) Analytics}
\label{fig:roi-analytics}
\end{figure}

This screen provides a post-session analytical view of where attention was
distributed in the classroom. Percentages for each region-of-interest are listed,
and an accompanying graph displays temporal patterns. Instructors may also toggle
visual elements such as scanpaths.

\subsection{10.6 Session Summary}

\begin{figure}[H]
\centering
\includegraphics[width=0.55\textwidth]{figure/Session_Summary.png}
\caption{Session Summary Screen}
\label{fig:session-summary}
\end{figure}

After a session ends, the summary view presents key metrics, including duration,
overall engagement, and the periods of highest attention. Instructors may export
a report for documentation or further analysis.

\subsection{10.7 Settings \& Calibration}

\begin{figure}[H]
\centering
\includegraphics[width=0.55\textwidth]{figure/Settings_Calibration.png}
\caption{Settings and Calibration Screen}
\label{fig:settings-calibration}
\end{figure}

The settings screen allows instructors to calibrate their eye-tracking device,
check battery level, test the camera feed, and adjust privacy or data storage
preferences. This screen ensures that sessions begin with reliable and
well-initialized hardware.

\section{Design of Communication Protocols}

This section describes the communication protocols used within \progname{}. 
Communication occurs between the client-facing dashboard, the real-time 
streaming backend, the analytics components, and the system’s secure storage 
infrastructure. All communication follows a modular service-based pattern 
where each module exposes a set of services that may be invoked by higher-level 
modules as shown in the uses hierarchy.

\subsection{11.1 Overall Architecture}

System communication is organized around three main channels:

\begin{itemize}
  \item \textbf{Real-Time Data Delivery:} Raw gaze streams and aggregated 
  attention metrics are transmitted from the Data Ingestion and Real-Time 
  Streaming modules to the Dashboard Visualization module.
  
  \item \textbf{Asynchronous Analytics Pipeline:} Preprocessed gaze data is 
  passed from the Data Preprocessing module to the Engagement Analytics and 
  Correlation \& Visual Analysis modules for metric generation.

  \item \textbf{Secure Storage and Retrieval:} Privacy \& Infrastructure 
  manages anonymized storage of both raw and processed data, making it 
  accessible to authorized modules such as Reporting.
\end{itemize}

All communication follows structured data formats, with components interacting 
through well-defined access routines specified in the MIS. Communication does 
not rely on shared global state; instead, modules exchange data through 
explicit service calls or via the streaming pipeline.

\subsection{11.2 Data Flow Patterns}

Communication follows two major patterns:

\begin{itemize}
  \item \textbf{Push-Based Real-Time Streaming:} The Real-Time Streaming module 
  (M3) pushes processed gaze data and attention indicators to the Dashboard 
  Visualization module (M4). This enables instructors to view attention trends 
  with minimal latency.

  \item \textbf{Pull-Based Analysis and Reporting:} The Engagement Analytics 
  module (M8), Correlation \& Visual Analysis module (M9), and Reporting module 
  (M5) request preprocessed or stored data from the Privacy \& Infrastructure 
  module (M7) and Data Preprocessing module (M6) as needed to compute summary 
  statistics or generate class reports.
\end{itemize}

\subsubsection*{Temporal Characteristics}

\begin{itemize}
  \item Real-time streams operate on short intervals (approximately 1–2 seconds).
  \item Analytics requests are event-driven rather than continuous.
  \item Reporting operations occur after class and may be batched.
\end{itemize}

\subsection{11.3 Message Types}

Modules exchange data using structured message formats. Typical messages include:

\begin{itemize}
  \item \textbf{GazeFrame:} Raw or preprocessed gaze coordinates, timestamps, 
  pupil size, and detection confidence.

  \item \textbf{AttentionSummary:} Aggregated per-region or per-student 
  attention metrics computed by the Engagement Analytics module.

  \item \textbf{CorrelationMap:} Data structures produced by the Correlation \& 
  Visual Analysis module describing gaze distribution heatmaps.

  \item \textbf{SessionRecord:} Metadata describing a class session, stored and 
  retrieved through the Privacy \& Infrastructure module.

  \item \textbf{ReportBundle:} Final compiled data provided to the Reporting 
  module for output formatting.
\end{itemize}

These message formats allow the system to maintain a separation of concerns: 
modules only require knowledge of the structure of the data they consume, 
not how it was generated.

\subsection{11.4 Privacy and Security Constraints}

All communication adheres to the privacy and security requirements outlined in 
the Hazard Analysis. Key protections include:

\begin{itemize}
  \item \textbf{Anonymization:} No personal identifiers are stored or transmitted 
  with gaze data.
  \item \textbf{Access Control:} Only authorized modules may request stored 
  data from M7 (Privacy \& Infrastructure).
  \item \textbf{Data Minimization:} Real-time streaming transmits only 
  aggregated or anonymized indicators rather than raw identifying visual data.
  \item \textbf{Integrity Checks:} Modules using stored datasets perform 
  checksum or timestamp validation to avoid stale or corrupted data.
\end{itemize}

\subsection{11.5 Design Considerations}

The communication protocol design prioritizes:

\begin{itemize}
  \item \textbf{Low Latency:} Real-time streaming ensures timely updates for 
  instructors during live sessions.
  \item \textbf{Scalability:} Analytics modules operate independently of the 
  real-time pipeline, allowing more complex computations without slowing down 
  the dashboard.
  \item \textbf{Modularity:} Modules communicate through stable abstractions 
  defined in the MIS, reducing interdependencies.
  \item \textbf{Reliability:} Data retrieval and reporting do not depend on 
  the continuous availability of real-time processes, enabling fault tolerance.
\end{itemize}

This design ensures consistent, secure, and efficient communication between 
modules in \progname{}, supporting real-time visualization, post-session 
analysis, and long-term data storage.

\section{Timeline}

This section outlines the implementation timeline for \progname{}.  
The schedule is organized according to the module hierarchy described in 
Section~\ref{SecMH}, with lower‐level modules developed first to enable 
higher‐level functionality. The work is divided into phases that reflect the 
system’s data pipeline: ingestion, preprocessing, analytics, and visualization.

\subsection*{Phase 1 — Foundations and Hardware Integration}
\begin{itemize}
  \item \textbf{M1: Hardware-Hiding Module}  
  Set up and test Pupil Labs Neon glasses, central camera, and instructor 
  workstation. Confirm data access points and ensure that device drivers and 
  calibration tools function reliably.

  \item \textbf{M2: Data Ingestion Module}  
  Implement raw gaze and video capture, device synchronization, and 
  time-stamped buffering. Establish initial interfaces for downstream modules.
\end{itemize}

\subsection*{Phase 2 — Data Processing and Infrastructure}
\begin{itemize}
  \item \textbf{M6: Data Preprocessing Module}  
  Develop filtering, noise reduction, coordinate normalization, and 
  outlier detection routines. Produce preprocessed frames for use by analytics 
  modules.

  \item \textbf{M7: Privacy \& Infrastructure Module}  
  Implement anonymization, secure data storage, access control, and 
  infrastructure services (session management, integrity checks).
\end{itemize}

\subsection*{Phase 3 — Real-Time and Analytical Capabilities}
\begin{itemize}
  \item \textbf{M3: Real-Time Streaming Module}  
  Construct the pipeline for sending processed gaze frames and attention signals 
  to the dashboard at 1–2 second intervals.

  \item \textbf{M8: Engagement Analytics Module}  
  Build algorithms for attention scoring, fixation detection, and aggregated 
  engagement metrics.

  \item \textbf{M9: Correlation \& Visual Analysis Module}  
  Implement spatial correlation calculations, heatmap generation, and visual 
  analysis tools using preprocessed data and analytics outputs.
\end{itemize}

\subsection*{Phase 4 — User-Facing Interfaces}
\begin{itemize}
  \item \textbf{M4: Dashboard Visualization Module}  
  Integrate real-time streaming, analytics, and correlation outputs into a 
  unified instructor-facing dashboard. Design displays for class-level and 
  region-level attention indicators.

  \item \textbf{M5: Reporting Module}  
  Generate post-session summaries that combine analytics and visualizations. 
  Produce exports or formatted reports based on session data.
\end{itemize}

\subsection*{Phase 5 — Integration, Testing, and Refinement}
\begin{itemize}
  \item \textbf{System Integration}  
  Combine all modules into an end-to-end pipeline from ingestion to reporting. 
  Ensure consistent data formats and stable communication across modules.

  \item \textbf{Verification and Validation}  
  Conduct unit tests, integration tests, and scenario-based validation 
  following the VnV Plan. Address feedback based on latency, accuracy, and 
  usability tests.

  \item \textbf{Performance and Privacy Review}  
  Verify that modules meet privacy constraints, maintain real-time performance 
  targets, and operate reliably under varying classroom conditions.

  \item \textbf{Final Documentation and Deliverables}  
  Prepare final reports, updated design documents, and demonstration artifacts 
  for Rev 1 and project completion.
\end{itemize}

This phased timeline ensures that core data-handling components are completed 
first, enabling analytics and visualization to be developed on a stable 
foundation. It also supports iterative testing and refinement as the pipeline 
becomes fully integrated.

\bibliographystyle {plainnat}
\bibliography{../../../refs/References}

\newpage{}

\end{document}
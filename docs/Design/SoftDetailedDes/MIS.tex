  \documentclass[12pt, titlepage]{article}

  \usepackage{amsmath, mathtools}

  \usepackage[round]{natbib}
  \usepackage{amsfonts}
  \usepackage{amssymb}
  \usepackage{graphicx}
  \usepackage{colortbl}
  \usepackage{xr}
  \usepackage{hyperref}
  \usepackage{longtable}
  \usepackage{xfrac}
  \usepackage{tabularx}
  \usepackage{float}
  \usepackage{siunitx}
  \usepackage{booktabs}
  \usepackage{multirow}
  \usepackage[section]{placeins}
  \usepackage{caption}
  \usepackage{fullpage}

  \hypersetup{
  bookmarks=true,     % show bookmarks bar?
  colorlinks=true,       % false: boxed links; true: colored links
  linkcolor=red,          % color of internal links (change box color with linkbordercolor)
  citecolor=blue,      % color of links to bibliography
  filecolor=magenta,  % color of file links
  urlcolor=cyan          % color of external links
  }

  \usepackage{array}

  \externaldocument{../../SRS/SRS}

  \input{../../Comments}
  %% Common Parts

\newcommand{\progname}{Software Engineering} % PUT YOUR PROGRAM NAME HERE
\newcommand{\authname}{Team 21, Visionaries
\\ Angela Zeng
\\ Ann Shi
\\ Ibrahim Sahi
\\ Manan Sharma
\\ Stanley Chen} % AUTHOR NAMES                  

\usepackage{hyperref}
    \hypersetup{colorlinks=true, linkcolor=blue, citecolor=blue, filecolor=blue,
                urlcolor=blue, unicode=false}
    \urlstyle{same}
                                


  \begin{document}

  \title{Module Interface Specification for \progname{}}

  \author{\authname}

  \date{\today}

  \maketitle

  \pagenumbering{roman}

  \section{Revision History}

  \begin{tabularx}{\textwidth}{p{3cm}p{2cm}X}
  \toprule {\bf Date} & {\bf Version} & {\bf Notes}\\
  \midrule
  Nov 13 & 1.0 & Version 1 completed\\
  \bottomrule
  \end{tabularx}

  ~\newpage

  \section{Symbols, Abbreviations and Acronyms}

  See SRS Documentation at \url{https://github.com/mansha71/CapstoneProject/tree/main/docs/SRS}

  \newpage

  \tableofcontents

  \newpage

  \pagenumbering{arabic}

  \section{Introduction}

  The following document details the Module Interface Specifications for Visionaries, a system designed to analyze and visualize student engagement during lectures using eye-tracking technology. Each module is described in terms of its purpose, syntax, semantics, and interactions with other modules.

  Complementary documents include the System Requirement Specifications
  and Module Guide.  The full documentation and implementation can be
  found at \url{https://github.com/mansha71/CapstoneProject}.

  \section{Notation}

  The structure of the MIS for modules comes from \citet{HoffmanAndStrooper1995},
  with the addition that template modules have been adapted from
  \cite{GhezziEtAl2003}.  The mathematical notation comes from Chapter 3 of
  \citet{HoffmanAndStrooper1995}.  For instance, the symbol := is used for a
  multiple assignment statement and conditional rules follow the form $(c_1
  \Rightarrow r_1 | c_2 \Rightarrow r_2 | ... | c_n \Rightarrow r_n )$.

  The following table summarizes the primitive data types used by \progname. 

  \begin{center}
  \renewcommand{\arraystretch}{1.2}
  \noindent 
  \begin{tabular}{l l p{7.5cm}} 
  \toprule 
  \textbf{Data Type} & \textbf{Notation} & \textbf{Description}\\ 
  \midrule
  character & char & a single symbol or digit\\
  integer & $\mathbb{Z}$ & a number without a fractional component in (-$\infty$, $\infty$) \\
  natural number & $\mathbb{N}$ & a number without a fractional component in [1, $\infty$) \\
  real & $\mathbb{R}$ & any number in (-$\infty$, $\infty$)\\
  \bottomrule
  \end{tabular} 
  \end{center}

  \noindent
  The specification of \progname \ uses some derived data types: sequences, strings, and
  tuples. Sequences are lists filled with elements of the same data type. Strings
  are sequences of characters. Tuples contain a list of values, potentially of
  different types. In addition, \progname \ uses functions, which
  are defined by the data types of their inputs and outputs. Local functions are
  described by giving their type signature followed by their specification.

  \section{Module Decomposition}

  The following table is taken directly from the Module Guide document for this project.

  \begin{table}[h!]
  \centering
  \begin{tabular}{p{0.3\textwidth} p{0.6\textwidth}}
  \toprule
  \textbf{Level 1} & \textbf{Level 2}\\
  \midrule

  {Hardware-Hiding} & Pupil Labs Neon Glasses, Central Camera, Instructor Laptop/Desktop \\
  \midrule

  \multirow{8}{0.3\textwidth}{Behaviour-Hiding} 
  & Data Ingestion Module\\
  & Real-Time Streaming Module\\
  & Data Preprocessing Module\\
  & Privacy \& Infrastructure Module\\
  & Engagement Analytics Module\\
  & Correlation \& Visual Analysis Module\\
  & Dashboard Visualization Module\\
  & Reporting Module\\
  \midrule

  \multirow{3}{0.3\textwidth}{Software Decision} 
  & Data Storage and API Layer\\
  & Authentication and Session Management\\
  & Visualization Frameworks and Libraries\\
  \bottomrule

  \end{tabular}
  \caption{Module Hierarchy for Visionaries System}
  \label{TblMH}
  \end{table}

  \newpage
  ~\newpage

  \section{MIS of Data Ingestion Module} \label{DataIngestionModule}

\subsection{Module}

This module entails collecting eye-tracking and video data from multiple eye-tracking devices, video data from the central camera and screen recording, as well as instructor audio recordings and pre-/post-lecture questionnaire responses. It is responsible solely for the reliable capture and initial storage of all incoming raw data streams, without performing any preprocessing or filtering operations.

\subsection{Uses}

Real-Time Streaming Module, Privacy \& Infrastructure Module

\subsection{Syntax}

\subsubsection{Exported Constants}

None.

\subsubsection{Exported Access Programs}

\begin{table}[h!]
\centering
\renewcommand{\arraystretch}{1.2}
\begin{tabular}{p{3.2cm} p{4.2cm} p{4.2cm} p{2.5cm}}
\toprule
\textbf{Name} & \textbf{Input} & \textbf{Output} & \textbf{Exceptions} \\
\midrule
collectGazeData & deviceID, streamConfig & rawGazeData & ConnectionError \\
collectVideoData & sourceID, streamConfig & rawVideoData & ConnectionError \\
collectAudioData & deviceID & rawAudioData & ConnectionError \\
collectQuestionnaire & formInput & questionnaireData & FormatError \\
storeRawData & dataBundle & status & IOError \\
\bottomrule
\end{tabular}
\caption{Exported Access Programs for Data Ingestion Module}
\label{tab:dataIngestionAP}
\end{table}

\subsection{Semantics}

\subsubsection{State Variables}

storedData: list of captured raw data streams currently saved in local or cloud storage.

\subsubsection{Environment Variables}

Connections to Pupil Labs Neon devices, the central camera, screen-recording source, audio devices, and questionnaire input sources.

\subsubsection{Assumptions}

All recording devices are properly connected, network latency is within acceptable limits for real-time transfer, and the file system or storage backend is accessible.

\subsubsection{Access Routine Semantics}

\noindent collectGazeData():
\begin{itemize}
\item transition: Initiates collection of gaze and world-view video data from the connected eye-tracking devices.
\item output: Returns raw gaze and eye-camera data.
\item exception: Raises ConnectionError if device is unavailable.
\end{itemize}

\noindent collectVideoData():
\begin{itemize}
\item transition: Records video data from the central camera and screen-capture feed.
\item output: Returns raw video frames.
\item exception: Raises ConnectionError if any video source is unavailable.
\end{itemize}

\noindent collectAudioData():
\begin{itemize}
\item transition: Captures instructor audio from connected microphone devices.
\item output: Returns raw audio stream data.
\item exception: Raises ConnectionError if the audio device is unavailable.
\end{itemize}

\noindent collectQuestionnaire():
\begin{itemize}
\item transition: Collects pre- or post-lecture questionnaire submissions.
\item output: Returns questionnaire data.
\item exception: Raises FormatError for incomplete or malformed responses.
\end{itemize}

\noindent storeRawData():
\begin{itemize}
\item transition: Saves all raw data streams to secure storage with metadata describing device, timestamp, and session.
\item output: Returns success status.
\item exception: Raises IOError if data cannot be written.
\end{itemize}

\subsubsection{Local Functions}

validateConnection(deviceID), verifySourceAvailability(sourceID)

  \newpage

  \section{MIS of Real-Time Streaming Module} \label{RealTimeStreamingModule}

  \subsection{Module}

  This module handles the continuous transmission of eye-tracking and video data to the dashboard for the live visualization. It ensures low-latency data flow and manages temporary buffering to keep the stream stable.

  \subsection{Uses}

  Data Ingestion Module, Data Preprocessing Module, Dashboard Visualization Module

  \subsection{Syntax}

  \subsubsection{Exported Constants}

  None.

  \subsubsection{Exported Access Programs}

  \begin{table}[h!]
  \centering
  \renewcommand{\arraystretch}{1.2}
  \begin{tabular}{p{3.2cm} p{4.2cm} p{4.2cm} p{2.5cm}}
  \toprule
  \textbf{Name} & \textbf{Input} & \textbf{Output} & \textbf{Exceptions} \\
  \midrule
  initializeStream & configSettings & streamSession & ConnectionError \\
  transmitData & processedData & status & StreamError \\
  terminateStream & sessionID & status & None \\
  \bottomrule
  \end{tabular}
  \caption{Exported Access Programs for Real-Time Streaming Module}
  \label{tab:realTimeStreamingAP}
  \end{table}


  \subsection{Semantics}

  \subsubsection{State Variables}

  activeStream: maintains the current streaming session status.

  \subsubsection{Environment Variables}

  Network socket for dashboard communication and connected video devices.

  \subsubsection{Assumptions}

  Stable network connection and active dashboard session are available.

  \subsubsection{Access Routine Semantics}

  \noindent initializeStream():
  \begin{itemize}
  \item transition: Opens a new connection to the dashboard and starts transmitting.
  \item output: Returns a session object.
  \item exception: Raises ConnectionError if the network fails.
  \end{itemize}

  \noindent transmitData():
  \begin{itemize}
  \item transition: Sends processed gaze and video frames in real time.
  \item output: Returns success confirmation.
  \item exception: Raises StreamError if data loss occurs.
  \end{itemize}

  \noindent terminateStream():
  \begin{itemize}
  \item transition: Ends the streaming session and clears buffers.
  \item output: Returns final session status.
  \item exception: None.
  \end{itemize}

  \subsubsection{Local Functions}

  bufferData(), reconnectStream()

  \newpage

 \section{MIS of Data Preprocessing Module} \label{DataPreprocessingModule}

\subsection{Module}

This module prepares the captured data for analysis by filtering noise, normalizing coordinates, and structuring the gaze and video data into usable formats. The recording inputs from the eye-tracking devices, central camera, and screen-recording coordinates are synchronized for homography, and instructor coordinates are generated throughout the recording. The screen-recording input is categorized based on visual item classes (e.g., text, diagram, image regions), and the audio stream is denoised, segmented, and temporally aligned with the synchronized video data. Questionnaire responses are cleaned by removing empty, duplicate, or invalid entries. This ensures all data sources are aligned, validated, and ready for analysis in downstream modules.

\subsection{Uses}

Data Ingestion Module, Real-Time Streaming Module, Privacy \& Infrastructure Module

\subsection{Syntax}

\subsubsection{Exported Constants}

None.

\subsubsection{Exported Access Programs}

\begin{table}[h!]
\centering
\renewcommand{\arraystretch}{1.2}
\begin{tabular}{p{3.8cm} p{4.2cm} p{4.2cm} p{2.5cm}}
\toprule
\textbf{Name} & \textbf{Input} & \textbf{Output} & \textbf{Exceptions} \\
\midrule
filterNoise & rawData & cleanData & DataError \\
synchronizeStreams & cleanDataBundle & syncedData & SyncError \\
computeHomography & syncedVideoData & alignedData & MathError \\
categorizeVisualItems & screenRecording & categorizedSegments & FormatError \\
cleanAudioStream & rawAudioData & processedAudio & AudioError \\
cleanQuestionnaireResponses & rawQuestionnaireData & validatedQuestionnaire & DataError \\
formatForAnalysis & alignedData & structuredData & FormatError \\
\bottomrule
\end{tabular}
\caption{Exported Access Programs for Data Preprocessing Module}
\label{tab:dataPreprocessingAP}
\end{table}

\subsection{Semantics}

\subsubsection{State Variables}

None. Operates statelessly on provided input data.

\subsubsection{Environment Variables}

None.

\subsubsection{Assumptions}

Input data follows the expected format, timestamps are available for all streams, and synchronization metadata is provided from the ingestion stage.

\subsubsection{Access Routine Semantics}

\noindent filterNoise():
\begin{itemize}
\item transition: Applies filters (e.g., smoothing, low-pass, temporal averaging) to remove noise and stabilize gaze, video, and audio streams.
\item output: Returns denoised data streams.
\item exception: Raises DataError for corrupted or incomplete data.
\end{itemize}

\noindent synchronizeStreams():
\begin{itemize}
\item transition: Aligns gaze, video, screen recording, and audio streams using timestamps or synchronization cues.
\item output: Returns synchronized multi-stream data bundle.
\item exception: Raises SyncError if synchronization fails.
\end{itemize}

\noindent computeHomography():
\begin{itemize}
\item transition: Computes the spatial mapping between gaze coordinates and screen coordinates via homography transformations.
\item output: Returns spatially aligned gaze data.
\item exception: Raises MathError if homography computation fails.
\end{itemize}

\noindent categorizeVisualItems():
\begin{itemize}
\item transition: Segments the screen recording into distinct visual item classes (text, diagram, etc.) using frame analysis.
\item output: Returns categorized video segments.
\item exception: Raises FormatError if visual item parsing fails.
\end{itemize}

\noindent cleanAudioStream():
\begin{itemize}
\item transition: Removes background noise, segments silence, and aligns cleaned audio with video frames.
\item output: Returns processed audio synchronized with visual data.
\item exception: Raises AudioError if audio cleaning fails.
\end{itemize}

\noindent cleanQuestionnaireResponses():
\begin{itemize}
\item transition: Removes incomplete, duplicate, or malformed questionnaire responses.
\item output: Returns validated questionnaire dataset.
\item exception: Raises DataError if questionnaire data cannot be validated.
\end{itemize}

\noindent formatForAnalysis():
\begin{itemize}
\item transition: Structures synchronized and aligned data into a standardized schema suitable for analytics and visualization modules.
\item output: Returns fully formatted data ready for downstream analysis.
\item exception: Raises FormatError for invalid schema generation.
\end{itemize}

\subsubsection{Local Functions}

interpolateMissingFrames(), alignTimestamps(), extractInstructorCoordinates(), computeVisualClasses()

  \newpage

  \section{MIS of Privacy and Infrastructure Module} \label{PrivacyInfrastructureModule}

  \subsection{Module}

  This module ensures secure data handling and stability of the system. It manages anonymization, encryption, and access control. It also handles maintaining infrastructure-level configurations and compliance with privacy policies. This includes securing instructor audio recordings and questionnaire response data through encryption and anonymization.

  \subsection{Uses}

  Data Ingestion Module, Data Preprocessing Module, Real-Time Streaming Module

  \subsection{Syntax}

  \subsubsection{Exported Constants}

  None.

  \subsubsection{Exported Access Programs}

  \begin{table}[h!]
  \centering
  \renewcommand{\arraystretch}{1.2}
  \begin{tabular}{p{3.8cm} p{4.2cm} p{4.2cm} p{2.5cm}}
  \toprule
  \textbf{Name} & \textbf{Input} & \textbf{Output} & \textbf{Exceptions} \\
  \midrule
  encryptData & plainData & encryptedData & EncryptionError \\
  anonymizeParticipant & userData & anonymizedData & PrivacyError \\
  validateAccess & userRole, resource & permissionStatus & AuthError \\
  monitorInfrastructure & metrics & systemReport & None \\
  \bottomrule
  \end{tabular}
  \caption{Exported Access Programs for Privacy and Infrastructure Module}
  \label{tab:privacyInfrastructureAP}
  \end{table}


  \subsection{Semantics}

  \subsubsection{State Variables}

  accessLog: records of authenticated access attempts and actions performed.

  \subsubsection{Environment Variables}

  System environment variables for encryption keys, network configurations, and role-based access rules.

  \subsubsection{Assumptions}

  All data passing through this module adheres to encryption and anonymization standards under PIPEDA.

  Audio data and questionnaire responses must also follow the system's encryption and anonymization requirements.

  \subsubsection{Access Routine Semantics}

  \noindent encryptData():
  \begin{itemize}
  \item transition: Applies encryption to sensitive data before storage or transmission.
  \item output: Returns encrypted data.
  \item exception: Raises EncryptionError if encryption fails.
  \end{itemize}

  \noindent anonymizeParticipant():
  \begin{itemize}
  \item transition: Removes personally identifiable information from datasets.
  \item output: Returns anonymized data.
  \item exception: Raises PrivacyError if anonymization is incomplete.
  \end{itemize}

  \noindent validateAccess():
  \begin{itemize}
  \item transition: Checks user permissions based on their assigned role.
  \item output: Returns access approval or denial.
  \item exception: Raises AuthError for invalid credentials.
  \end{itemize}

  \noindent monitorInfrastructure():
  \begin{itemize}
  \item transition: Tracks system metrics and logs anomalies for administrative review.
  \item output: Returns current system status.
  \item exception: None.
  \end{itemize}

  \subsubsection{Local Functions}

  generateKey(), rotateLogs(), alertAdmin()

  \newpage

  \section{MIS of Engagement Analytics Module} \label{EngagementAnalyticsModule}

  \subsection{Module}

  This module analyzes student engagement by combining attention metrics with pre- and post-lecture questionnaire results. Questionnaire data is collected through the Data Ingestion Module, anonymized by the Privacy \& Infrastructure Module and preprocessed before reaching this module. Each question is associated with the content of a specific slide, helping compute slide-level learning scores. With pre- and post-lecture questionnaires, changes in student understanding can be identified and correlated with gaze-data. These analytics quantify the effectiveness of each slide, object type, instructor-directed attention, and instructor audio trends.

  \subsection{Uses}

  Data Preprocessing Module, Privacy \& Infrastructure Module

  \subsection{Syntax}

  \subsubsection{Exported Constants}

  None.

  \subsubsection{Exported Access Programs}

  \begin{table}[h!]
  \centering
  \renewcommand{\arraystretch}{1.2}
  \begin{tabular}{p{3.2cm} p{4.2cm} p{4.2cm} p{2.5cm}}
  \toprule
  \textbf{Name} & \textbf{Input} & \textbf{Output} & \textbf{Exceptions} \\
  \midrule
  computeLearningScores & anonymizedQuestionnaireData, slideMap & learningScores & DataError \\
  \bottomrule
  \end{tabular}
  \caption{Exported Access Programs for Engagement Analytics Module}
  \label{tab:engagementAnalyticsAP}
  \end{table}

  \subsection{Semantics}

  \subsubsection{State Variables}

  None.

  \subsubsection{Environment Variables}

  None.

  \subsubsection{Assumptions}

  Questionnaire data has already been collected, anonymized, and validated.  
  Each questionnaire item includes a slide reference.

  \subsubsection{Access Routine Semantics}

  \noindent computeLearningScores():
  \begin{itemize}
  \item transition: Computes pre/post learning gains per slide using questionnaire data linked to slide identifiers. Aggregates multiple questions associated with the same slide.
  \item output: Returns learning scores per slide and overall session learning outcome scores.
  \item exception: Raises DataError for missing, malformed, or improperly mapped questionnaire items.
  \end{itemize}

  \subsubsection{Local Functions}

  computeLearningDelta(), aggregateSlideScores()

  \newpage

  \section{MIS of Correlation and Visual Analysis Module} \label{CorrelationVisualAnalysisModule}

  \subsection{Module}

  This module maps world-view gaze coordinates into screen-space coordinates and instructor-region coordinates. It identifies slide transitions, classifies gaze targets, and labels all gaze data with object-level or instructor-region tags. It also aligns learning scores from the Engagement Analytics Module with each slide for downstream visualization and reporting.

  \subsection{Uses}

  Data Preprocessing Module, Engagement Analytics Module

  \subsection{Syntax}

  \subsubsection{Exported Constants}

  None.

  \subsubsection{Exported Access Programs}

  \begin{table}[h!]
  \centering
  \renewcommand{\arraystretch}{1.2}
  \begin{tabular}{p{3.2cm} p{4.2cm} p{4.2cm} p{2.5cm}}
  \toprule
  \textbf{Name} & \textbf{Input} & \textbf{Output} & \textbf{Exceptions} \\
  \midrule
  mapCoordinatesToScene & worldViewData, sceneModel & mappedData & MappingError \\
  detectSlideTransitions & videoStream & slideTimeline & VideoError \\
  classifyGazeTarget & mappedData, slideObjects, instructorCoords & labeledGazeData & ClassificationError \\
  alignLearningScores & learningScores, slideTimeline & slideLearningMap & None \\
  \bottomrule
  \end{tabular}
  \caption{Exported Access Programs for Correlation and Visual Analysis Module}
  \label{tab:correlationVisualAnalysisAP}
  \end{table}

  \subsection{Semantics}

  \subsubsection{State Variables}

  None.

  \subsubsection{Environment Variables}

  None.

  \subsubsection{Assumptions}

  Instructor coordinates and slide-object metadata are available.  
  Learning scores reference valid slide identifiers.

  \subsubsection{Access Routine Semantics}

  \noindent mapCoordinatesToScene():
  \begin{itemize}
  \item transition: Converts world-view gaze vectors into screen-plane or instructor-region coordinates.
  \item output: Returns mapped gaze data relative to the scene.
  \item exception: Raises MappingError if projection fails.
  \end{itemize}

  \noindent detectSlideTransitions():
  \begin{itemize}
  \item transition: Detects slide changes from video or metadata.
  \item output: Returns a timeline of slide transitions.
  \item exception: Raises VideoError for detection failures.
  \end{itemize}

  \noindent classifyGazeTarget():
  \begin{itemize}
  \item transition: Classifies each gaze point as directed at a screen object, the instructor region, or off-screen.
  \item output: Returns labeled gaze data.
  \item exception: Raises ClassificationError for ambiguous or invalid mappings.
  \end{itemize}

  \noindent alignLearningScores():
  \begin{itemize}
  \item transition: Associates each slide with its corresponding learning outcome scores.
  \item output: Returns a mapping of slide IDs to learning metrics.
  \item exception: None.
  \end{itemize}

  \subsubsection{Local Functions}

  projectToPlane(), identifyObjects(), mergeWithLearningData()

  \newpage

  \section{MIS of Dashboard Visualization Module} \label{DashboardVisualizationModule}

  \subsection{Module}

  This module visualizes gaze, focus, instructor-region attention, object-level attention, and learning outcome scores. It provides a synchronized playback interface that overlays heatmaps, slide boundaries, object segments, and learning scores. The dashboard supports multiple viewing modes, including world-view playback, symbolic screen view, and object-specific focus reconstruction.

  \subsection{Uses}

  Real-Time Streaming Module, Correlation \& Visual Analysis Module, Engagement Analytics Module

  \subsection{Syntax}

  \subsubsection{Exported Constants}

  None.

  \subsubsection{Exported Access Programs}

  \begin{table}[h!]
  \centering
  \renewcommand{\arraystretch}{1.2}
  \begin{tabular}{p{3.2cm} p{4.2cm} p{4.2cm} p{2.5cm}}
  \toprule
  \textbf{Name} & \textbf{Input} & \textbf{Output} & \textbf{Exceptions} \\
  \midrule
  renderPlayer & playbackData & playerUI & RenderError \\
  generateHeatmap & gazeData, mode & heatmap & VisualizationError \\
  displayObjectStats & labeledGazeData & statsView & None \\
  showLearningScores & slideLearningMap & learningPanel & None \\
  segmentPlayback & segmentationRules & segmentedTimeline & None \\
  \bottomrule
  \end{tabular}
  \caption{Exported Access Programs for Dashboard Visualization Module}
  \label{tab:dashboardVisualizationAP}
  \end{table}

  \subsection{Semantics}

  \subsubsection{State Variables}

  activeView: currently selected visualization mode.

  \subsubsection{Environment Variables}

  Rendering and graphical libraries within the dashboard environment.

  \subsubsection{Assumptions}

  All gaze, slide, instructor, and learning score metadata is synchronized.

  \subsubsection{Access Routine Semantics}

  \noindent renderPlayer():
  \begin{itemize}
  \item transition: Displays synchronized playback with audio, heatmaps, symbolic overlays, and slide markers.
  \item output: Returns a player UI instance.
  \item exception: Raises RenderError if visual components fail to load.
  \end{itemize}

  \noindent generateHeatmap():
  \begin{itemize}
  \item transition: Produces continuous or object-discrete heatmaps.
  \item output: Returns a heatmap image or overlay.
  \item exception: Raises VisualizationError if heatmap creation fails.
  \end{itemize}

  \noindent displayObjectStats():
  \begin{itemize}
  \item transition: Displays percentage of gaze directed at each object and the instructor region.
  \item output: Returns a statistical summary panel.
  \item exception: None.
  \end{itemize}

  \noindent showLearningScores():
  \begin{itemize}
  \item transition: Displays slide-level learning outcome scores alongside the playback timeline.
  \item output: Returns a learning score panel or overlay.
  \item exception: None.
  \end{itemize}

  \noindent segmentPlayback():
  \begin{itemize}
  \item transition: Splits playback based on audio cues, slide boundaries, or focus patterns.
  \item output: Segmented playback timeline.
  \item exception: None.
  \end{itemize}

  \subsubsection{Local Functions}

  buildSlidePanels(), computeDiscreteFocus(), overlayLearningIndicators()

  \newpage

  \section{MIS of Reporting Module} \label{ReportingModule}

  \subsection{Module}

  This module generates automated or customized reports integrating object-level attention, instructor-region attention, slide timelines, and learning outcome scores. Reports summarize both attention patterns and learning gains, enabling instructors to evaluate instructional effectiveness.

  \subsection{Uses}

  Engagement Analytics Module, Dashboard Visualization Module

  \subsection{Syntax}

  \subsubsection{Exported Constants}

  None.

  \subsubsection{Exported Access Programs}

  \begin{table}[h!]
  \centering
  \renewcommand{\arraystretch}{1.2}
  \begin{tabular}{p{3.2cm} p{4.2cm} p{4.2cm} p{2.5cm}}
  \toprule
  \textbf{Name} & \textbf{Input} & \textbf{Output} & \textbf{Exceptions} \\
  \midrule
  generateAutoReport & analyticsData, slideLearningMap & reportDocument & ReportError \\
  customizeReport & selectionCriteria, analyticsData, slideLearningMap & customReport & None \\
  exportReport & reportDocument, format & fileOutput & FileError \\
  \bottomrule
  \end{tabular}
  \caption{Exported Access Programs for Reporting Module}
  \label{tab:reportingAP}
  \end{table}

  \subsection{Semantics}

  \subsubsection{State Variables}

  None.

  \subsubsection{Environment Variables}

  Access to file storage or cloud-based export locations.

  \subsubsection{Assumptions}

  All analytics and learning score data inputs are precomputed and validated.

  \subsubsection{Access Routine Semantics}

  \noindent generateAutoReport():
  \begin{itemize}
  \item transition: Produces a full report with aggregated attention graphs, instructor-region attention, slide-object attention, and learning outcome scores.
  \item output: Returns a comprehensive report document.
  \item exception: Raises ReportError if report generation fails.
  \end{itemize}

  \noindent customizeReport():
  \begin{itemize}
  \item transition: Creates a report for selected slides, objects, audio portions, or instructor-region attention metrics.
  \item output: Returns a customized report document.
  \item exception: None.
  \end{itemize}

  \noindent exportReport():
  \begin{itemize}
  \item transition: Converts the report to the specified format (PDF, HTML, etc.) and writes it to storage.
  \item output: Returns the exported file.
  \item exception: Raises FileError if export fails.
  \end{itemize}

  \subsubsection{Local Functions}

  plotLearningVsAttention(), compileSlideSections(), exportFile()

  \newpage

  \section*{Appendix --- Reflection}

  The information in this section will be used to evaluate the team members on the
  graduate attribute of Problem Analysis and Design.

  \input{../../Reflection.tex}

  \begin{enumerate}
    \item What went well while writing this deliverable? 

  \textbf{Stanley:} Once we settled on the module list, filling in the MIS felt pretty natural and I could lean on the SRS and MG a lot.

  \textbf{Manan:} The team was able to work well together and we were able to split the work effectively.

  \textbf{Angela:} We were able to meet with the supervisors this week and get a good grasp on what they're looking for in terms of the designing of our modules, dashboard, and eventually, POC. 

  \textbf{Ann:} During this deliverable we got to have another meeting with our supervisors, as well as meet the PhD students who developed SocialEyes. We got to learn more about the SocialEyes framework and see how the different modules are used.

  \textbf{Ibrahim:} Breaking the project into modules gave us a good understanding of what our next steps should be, and helped us plan ahead.

    \item What pain points did you experience during this deliverable, and how
      did you resolve them?

  \textbf{Stanley:} I struggled a bit with how detailed each interface should be, but looking at past MIS examples helped me find the right level.

  \textbf{Manan:} The main pain point we experienced was coordinating with our supervisors and coming up with a list of modules we though worked well with out system.

  \textbf{Angela:} While we were able to derive a sufficient list of modules for our system, it did take a lot of back-and-forth, as well as brainstorming to come up with the modules, the purpose for them, etc.

  \textbf{Ann:} Given the amount of functional and non-functional requirements listed in the SRS, I had to filter the amount that would be mapped to our modules. It was difficult for me to come up with the top and most appropriate requirements to choose in the traceability matrix. 

  \textbf{Ibrahim:} Determining how to break the project into modules was difficult at first, as we were unsure of the exact expectations of the supervisors for each stream of the project we were planning to work on.

    \item Which of your design decisions stemmed from speaking to your client(s)
    or a proxy (e.g. your peers, stakeholders, potential users)? For those that
    were not, why, and where did they come from?

  \textbf{Stanley:} Most of my decisions came from our meetings with the supervisors, especially their focus on making the tool genuinely useful for instructors.

  \textbf{Manan:} Decisions regarding the modules were made after speaking with our TA and supervisor to ensure that we were on the right track. We were suggested to add more modules to better break down the system and also focus on the processing aspect of the system.

  \textbf{Angela:} We were able to speak with both supervisors, as well as another pHD student who has experience with attention, to infer our design decisions. It really helped as well that the supervisors are not stranger's to teaching, so they're sort of the primary audience we'd be creating our modules for.

  \textbf{Ann:} Our design decisions for the modules were inspired from our meetings with our supervisors as well as looking at the current implementation of modules and the design of the already existing SocialEyes framework.

  \textbf{Ibrahim:} The decision to include modules for the Real-Time system and the Engagement section were made following a meeting with the supervisors.

    \item While creating the design doc, what parts of your other documents (e.g.
    requirements, hazard analysis, etc), it any, needed to be changed, and why?

  \textbf{Stanley:} Nothing major needed changing, but the MIS did show a few spots in the SRS where we could clarify real-time versus post-session features later.

  \textbf{Manan:} Nothing significantly changed for over previous documents as we had a solid foundation from our SRS and other documents.

  \textbf{Angela:} There isn't really anything that needs to be changed stemming just from the design doc. However, that will likely change based on feedback from supervisors, TA, and peer reviews.

  \textbf{Ann:} Nothing as of yet needed to be changed.

  \textbf{Ibrahim:} There were no significant changes required.

    \item What are the limitations of your solution?  Put another way, given
    unlimited resources, what could you do to make the project better? (LO\_ProbSolutions)

  \textbf{Stanley:} We’re limited by hardware access and data, so with more resources I’d want more devices, more classrooms, and a more polished, customizable dashboard.

  \textbf{Manan:} We are limited by the availability of eye-tracking devices and the accuracy of those devices. Being able to have these devices on hand at all times would make it easier to plan a system that works well with the hardware.

  \textbf{Angela:} We don't have free reign access over using the eye-tracking devices, so we'll have to plan with the supervisors accordingly to use them.

  \textbf{Ann:} There are multiple stretch goals and stretch functional requirements as listed in the SRS that we would love to tackle if we were given more time (privacy module, scalability, etc.)

  \textbf{Ibrahim:} The number of eye-tracking devices available, as well the logistical challenges of setting up the lectures limits the amount of data we can test on.

    \item Give a brief overview of other design solutions you considered.  What
    are the benefits and tradeoffs of those other designs compared with the chosen
    design?  From all the potential options, why did you select the documented design?
    (LO\_Explores)

  \textbf{Stanley:} We briefly considered a more monolithic design, but the current modular breakdown felt cleaner and easier to extend without rewriting everything.

  \textbf{Manan:} We considered breaking down the modules further but we felt that the current breakdown was sufficient to cover all aspects of the system without overcomplicating it. The tradeoff with having too many modules is that it can make the system harder to manage and understand.

  \textbf{Angela:} We briefly considered both a more monolithic design and a more fine-grained module breakdown, but chose the documented design because it provides clearer information hiding and better aligns with the SRS without adding unnecessary complexity.

  \textbf{Ann:} We considered further breaking down our current module design smaller, but were worried about adding more complexity to the project. We wanted to carefully chooose our modules for best mapping to our requirements as defined in the SRS.

  \textbf{Ibrahim:} We considered separating some of the modules into versions for post-session and real-time analysis, but ultimately decided that the underlying frameworks would similar and scaleable enough to not have to work on them completely separately.

  \end{enumerate}


  \end{document}

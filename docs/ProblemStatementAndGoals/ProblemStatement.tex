\documentclass{article}

\usepackage{tabularx}
\usepackage{booktabs}

\title{Problem Statement and Goals\\\progname}

\author{\authname}

\date{}

\input{../Comments}
%% Common Parts

\newcommand{\progname}{Software Engineering} % PUT YOUR PROGRAM NAME HERE
\newcommand{\authname}{Team 21, Visionaries
\\ Angela Zeng
\\ Ann Shi
\\ Ibrahim Sahi
\\ Manan Sharma
\\ Stanley Chen} % AUTHOR NAMES                  

\usepackage{hyperref}
    \hypersetup{colorlinks=true, linkcolor=blue, citecolor=blue, filecolor=blue,
                urlcolor=blue, unicode=false}
    \urlstyle{same}
                                


\begin{document}

\maketitle

\begin{table}[hp]
\caption{Revision History} \label{TblRevisionHistory}
\begin{tabularx}{\textwidth}{llX}
\toprule
\textbf{Date} & \textbf{Developer(s)} & \textbf{Change}\\
\midrule
September 23 & Ann & Updates to problem statement\\
Date2 & Name(s) & Description of changes\\
... & ... & ...\\
\bottomrule
\end{tabularx}
\end{table}

\section{Problem Statement}
Instructors lack access to real-time insights about where students direct their attention during learning activities, particularly in large-group settings. Without this data, it's hard to assess engagement, monitor collaboration, and adapt teaching strategies to improve the effectiveness of learning. Currently, there is an existing software framework (SocialEyes) used to scale single eye-tracking to multi-person social settings. However, there is a need for a system that can capture and analyze this group gaze data real-time during classroom activities, so that instructors better understand and respond to student attention and engagement in both synchronous and asynchronous learning contexts. 

\subsection{Problem}

\subsection{Inputs and Outputs}

\wss{Characterize the problem in terms of ``high level'' inputs and outputs.  
Use abstraction so that you can avoid details.}

\subsection{Stakeholders}

\subsection{Environment}

\wss{Hardware and Software Environment}

\section{Goals}

\section{Stretch Goals}

\section{Extras}

\wss{For CAS 741: State whether the project is a research project. This
designation, with the approval (or request) of the instructor, can be modified
over the course of the term.}

\wss{For SE Capstone: List your extras.  Potential extras include usability
testing, code walkthroughs, user documentation, formal proof, GenderMag
personas, Design Thinking, etc.  (The full list is on the course outline and in
Lecture 02.) Normally the number of extras will be two.  Approval of the extras
will be part of the discussion with the instructor for approving the project.
The extras, with the approval (or request) of the instructor, can be modified
over the course of the term.}

\newpage{}

\section*{Appendix --- Reflection}

\wss{Not required for CAS 741}

\input{../Reflection.tex}

\begin{enumerate}
    \item What went well while writing this deliverable? 
    
Stanley: Writing this deliverable went smoothly for me. The reason for this is that I was responsible for creating the team charter (which is on the development plan document), which was a simple process where I came up with some reasonable responses and checked in to see if my team agreed. 

    \item What pain points did you experience during this deliverable, and how
    did you resolve them?

Stanley: The only pain point that I experienced was making sure that everyone agreed with the expectations that I wrote (on the development plan document). One member said that if we should let our group know in advance that we will not be able to attend a meeting, that should be reasonable enough for him/her to not show up, which I updated in the expectations later.  

    \item How did you and your team adjust the scope of your goals to ensure
    they are suitable for a Capstone project (not overly ambitious but also of
    appropriate complexity for a senior design project)?

Stanley: For the most part, we met up with our instructors and team members to discuss how much time and effort we could afford to invest. Through this, we were able to gauge exactly the scope of our project.  

\end{enumerate}  

\end{document}
\documentclass{article}

\usepackage{tabularx}
\usepackage{booktabs}

\title{Problem Statement and Goals\\\progname}

\author{\authname}

\date{}

\input{../Comments}
%% Common Parts

\newcommand{\progname}{Software Engineering} % PUT YOUR PROGRAM NAME HERE
\newcommand{\authname}{Team 21, Visionaries
\\ Angela Zeng
\\ Ann Shi
\\ Ibrahim Sahi
\\ Manan Sharma
\\ Stanley Chen} % AUTHOR NAMES                  

\usepackage{hyperref}
    \hypersetup{colorlinks=true, linkcolor=blue, citecolor=blue, filecolor=blue,
                urlcolor=blue, unicode=false}
    \urlstyle{same}
                                


\begin{document}

\maketitle

\begin{table}[hp]
\caption{Revision History} \label{TblRevisionHistory}
\begin{tabularx}{\textwidth}{llX}
\toprule
\textbf{Date} & \textbf{Developer(s)} & \textbf{Change}\\
\midrule
09/23/2025 & Angela Zeng & Add first draft of problem statement and goals\\
Date2 & Name(s) & Description of changes\\
... & ... & ...\\
\bottomrule
\end{tabularx}
\end{table}

\section{Problem Statement}

\subsection{Problem}

Instructors lack access to real-time insights about where students direct their 
attention during learning activities, particularly in large-group settings. Without 
this data, it is difficult to assess engagement, monitor collaboration, and adapt 
teaching strategies to improve the effectiveness of learning. 

There is a need for a system that can capture and analyze group gaze data during 
classroom activities, so that instructors can better understand and respond to 
student attention and engagement in both synchronous and asynchronous 
learning contexts.

\subsection{Inputs and Outputs}

\wss{Characterize the problem in terms of ``high level'' inputs and outputs.  
Use abstraction so that you can avoid details.}

\subsection{Stakeholders}

\subsubsection*{Direct Stakeholders}

\begin{enumerate}
    \item Students
    \begin{itemize}
        \item Their gaze data is collected during learning activities.
        \item They benefit from potentially improved engagement and teaching methods.
    \end{itemize}

    \item Instructors / Professors
    \begin{itemize}
        \item Use gaze-based analytics and dashboards to adapt teaching.
        \item Participate in the research study and provide feedback.
    \end{itemize}

    \item Capstone Development Team
    \begin{itemize}
        \item Designs, builds, and tests the integrated learning platform.
        \item Works on solving technical challenges (e.g., real-time visualization, group data capture).
    \end{itemize}

    \item Researchers / Educational Technologists
    \begin{itemize}
        \item Analyze gaze data to study engagement and collaboration.
        \item Derive insights that inform system improvements and pedagogy.
    \end{itemize}
\end{enumerate}

\subsubsection*{Indirect Stakeholders}

\begin{enumerate}
    \item Future Students and Instructors
    \begin{itemize}
        \item Benefit from refined teaching methods and improved learning environments informed by this research.
    \end{itemize}

    \item Industry Partners / EdTech Companies
    \begin{itemize}
        \item Could leverage findings for commercial tools (e.g., learning analytics platforms).
    \end{itemize}

    \item University Administration (McMaster)
    \begin{itemize}
        \item Gains insights into teaching effectiveness and innovations in classroom technology.
        \item May decide on scaling or adopting such systems institution-wide.
    \end{itemize}
\end{enumerate}

\subsection{Environment}

\subsubsection*{Software Environment}

\begin{itemize}
    \item \textbf{Version Control and Collaboration:} GitHub will be used for source code management, issue tracking, project boards, and continuous integration/continuous delivery (CI/CD).
    \item \textbf{Integrated Development Environment (IDE):} Visual Studio Code (VS Code) will serve as the primary IDE.
\end{itemize}

\section{Goals}

\begin{enumerate}
    \item Develop a Learning Platform
    \begin{itemize}
        \item Integrate large-group eye tracking into both synchronous (live classes) 
        and asynchronous (recorded or online activities) learning contexts.
    \end{itemize}

    \item Enable Contextual Data Capture
    \begin{itemize}
        \item Log gaze data alongside classroom learning activities such as passive 
        content viewing (e.g., watching videos) and active group work 
        (e.g., exercises, discussions).
    \end{itemize}

    \item Conduct In-Person Research
    \begin{itemize}
        \item Run a study with instructors and students at McMaster University 
        to evaluate how gaze-based insights affect teaching and learning.
    \end{itemize}

    \item Inform Future System Designs
    \begin{itemize}
        \item Use study findings to guide the development of features like 
        instructor dashboards and real-time gaze analytics.
    \end{itemize}

    \item Tackle Key Technical Challenges
    \begin{itemize}
        \item Address issues in system design, reliable capture of group gaze data, 
        and effective real-time visualization of attention patterns.
    \end{itemize}
\end{enumerate}

\section{Stretch Goals}

\begin{enumerate}
    \item Support more complex classroom activities
    \begin{itemize}
        \item Extend the system beyond single-board, lecture-style classes to settings with multiple focal points (for example, group discussions or multiple boards), so it can handle a wider variety of learning environments.
    \end{itemize}

    \item Enhance real-time analytics
    \begin{itemize}
        \item Improve the live feedback available to instructors by going beyond simple engagement markers, while keeping the analytics lightweight enough to scale across many devices.
    \end{itemize}

    \item Strengthen privacy protections
    \begin{itemize}
        \item Ensure the system does not expose sensitive or personally identifiable information. This includes hiding personal details captured by eye cameras and preventing raw video feeds from being directly viewed by researchers or instructors.
    \end{itemize}
\end{enumerate}

\section{Extras}

\wss{For CAS 741: State whether the project is a research project. This
designation, with the approval (or request) of the instructor, can be modified
over the course of the term.}

\wss{For SE Capstone: List your extras.  Potential extras include usability
testing, code walkthroughs, user documentation, formal proof, GenderMag
personas, Design Thinking, etc.  (The full list is on the course outline and in
Lecture 02.) Normally the number of extras will be two.  Approval of the extras
will be part of the discussion with the instructor for approving the project.
The extras, with the approval (or request) of the instructor, can be modified
over the course of the term.}



\newpage{}

\section*{Appendix --- Reflection}

\textbf{Angela Zeng}
\begin{enumerate}
    \item After meeting with our supervisors and discussing the plan as a team, we were able to clarify and finalize the details for this deliverable. We also received supporting resources, including research papers and the GitHub repository, which helped us better understand the project requirements and identify the stakeholders it affects.  

    \item Since our supervisor was assigned closer to the deadline for this deliverable, we needed to accelerate our work and quickly build a strong understanding of the project. We resolved this challenge by holding an in-depth meeting with our supervisors, where we clarified expectations and addressed the sections of the deliverable that were initially unclear.  

    \item Initially, we drafted our goals based on assumptions drawn from the potential projects PDF. We later refined these goals by incorporating the rubric, lecture slides, and supervisor feedback, ensuring that they are both realistic in scope and appropriately complex for a Capstone project.  
\end{enumerate}

\end{document}
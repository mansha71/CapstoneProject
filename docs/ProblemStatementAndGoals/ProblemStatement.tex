\documentclass{article}

\usepackage{tabularx}
\usepackage{booktabs}

\title{Problem Statement and Goals\\\progname}

\author{\authname}

\date{}

%% Comments

\usepackage{color}

\newif\ifcomments\commentstrue %displays comments
%\newif\ifcomments\commentsfalse %so that comments do not display

\ifcomments
\newcommand{\authornote}[3]{\textcolor{#1}{[#3 ---#2]}}
\newcommand{\todo}[1]{\textcolor{red}{[TODO: #1]}}
\else
\newcommand{\authornote}[3]{}
\newcommand{\todo}[1]{}
\fi

\newcommand{\wss}[1]{\authornote{magenta}{SS}{#1}} 
\newcommand{\plt}[1]{\authornote{cyan}{TPLT}{#1}} %For explanation of the template
\newcommand{\an}[1]{\authornote{cyan}{Author}{#1}}

%% Common Parts

\newcommand{\progname}{Software Engineering} % PUT YOUR PROGRAM NAME HERE
\newcommand{\authname}{Team 21, Visionaries
\\ Ahmed Sahi
\\ Angela Zeng
\\ Ann Shi
\\ Manan Sharma
\\ Stanley Chen} % AUTHOR NAMES                  

\usepackage{hyperref}
    \hypersetup{colorlinks=true, linkcolor=blue, citecolor=blue, filecolor=blue,
                urlcolor=blue, unicode=false}
    \urlstyle{same}
                                


\begin{document}

\maketitle

\begin{table}[hp]
\caption{Revision History} \label{TblRevisionHistory}
\begin{tabularx}{\textwidth}{llX}
\toprule
\textbf{Date} & \textbf{Developer(s)} & \textbf{Change}\\
\midrule
09/23/2025 & All & Add first draft of problem statement and goals\\
09/23/2025 & Ann & Updates to problem statement, input, outputs, goals, extras\\
... & ... & ...\\
\bottomrule
\end{tabularx}
\end{table}

\section{Problem Statement}
\subsection{Problem}
Instructors lack access to real-time insights about where students direct their attention during learning activities, particularly in large-group settings. Without this data, it's hard to assess engagement, monitor collaboration, and adapt teaching strategies to improve the effectiveness of learning. Currently, there is an existing software framework (SocialEyes) used to scale single eye-tracking to multi-person social settings. However, there is a need for a system that can capture and analyze this group gaze data real-time during classroom activities, so that instructors better understand and respond to student attention and engagement in both synchronous and asynchronous learning contexts. 

\subsection{Inputs and Outputs}
Inputs 
\begin{enumerate}
    \item Aggregated eye-tracking gaze data from a group of people via the wearable Pupil Labs Neon device. This data includes: 
    \begin{itemize}
        \item Eye images with a resolution of 192 x 192 pixels and a frequency of 200 Hz 
        \item Egoview captured by a front facing camera, with a resolution of 1600 × 1200 pixels at 30 Hz
    \end{itemize}
    \item Other measurements of students’ learning outcomes (e.g: qualitative data obtained through surveys) 
    \begin{itemize}
        \item This is an additional consideration in terms of the inputs to the project and might or might not be used in the final project.  
    \end{itemize}
\end{enumerate}  
Outputs
\begin{enumerate}
    \item Instructor dashboard cleanly displaying group eye-tracking analytics for analysis
        \begin{itemize}
            \item Post-learning session analytics 
            \item Real-time analytics
            \item Additional learning measurements (qualitative/quantitative data collected by students on their experience during the session)
        \end{itemize}
\end{enumerate}  

\subsection{Stakeholders}

\subsubsection*{Direct Stakeholders}

\begin{enumerate}
    \item Students
    \begin{itemize}
        \item Their gaze data is collected during learning activities.
        \item They benefit from potentially improved engagement and teaching methods.
    \end{itemize}

    \item Instructors / Professors
    \begin{itemize}
        \item Use gaze-based analytics and dashboards to adapt teaching.
        \item Participate in the research study and provide feedback.
    \end{itemize}

    \item Researchers / Educational Technologists
    \begin{itemize}
        \item Analyze gaze data to study engagement and collaboration.
        \item Derive insights that inform system improvements and pedagogy.
    \end{itemize}
\end{enumerate}

\subsubsection*{Indirect Stakeholders}

\begin{enumerate}
    \item Future Students and Instructors
    \begin{itemize}
        \item Benefit from refined teaching methods and improved learning environments informed by this research.
    \end{itemize}

    \item Industry Partners / EdTech Companies
    \begin{itemize}
        \item Could leverage findings for commercial tools (e.g., learning analytics platforms).
    \end{itemize}

    \item University Administration (McMaster)
    \begin{itemize}
        \item Gains insights into teaching effectiveness and innovations in classroom technology.
        \item May decide on scaling or adopting such systems institution-wide.
    \end{itemize}
\end{enumerate}

\subsection{Environment}

\subsubsection*{Software Environment}

\begin{itemize}
    \item \textbf{Version Control and Collaboration:} GitHub will be used for source code management, issue tracking, project boards, and continuous integration/continuous delivery (CI/CD).
    \item \textbf{Integrated Development Environment (IDE):} Visual Studio Code (VS Code) will serve as the primary IDE.
\end{itemize}

\section{Goals}

  \begin{enumerate}
  \item Adapting prior framework (SocialEyes) setup for data collection in the learning context and generate post-session analytics
  \item Support real-time analytics 
    \begin{itemize}
        \item Finding a light-weight model/algorithm used to generate real-time analytics
        \item Determine the most important analytics to generate real-time
    \end{itemize}
    \item Building a instructor dashboard
        \begin{itemize}
            \item Displaying real-time and post-session analytics for instructors to gain insight of learning activities in the classroom
               \item Account and roles within dashboard
                    \begin{itemize}
                        \item Role-based access and account system for instructors (other roles may be added later)
                        \item Eye-tracker data can be identified with each individual student (each device has a unique IP address)
                    \end{itemize}
        \end{itemize}
  \end{enumerate}

\section{Stretch Goals}

\begin{enumerate}
    \item Support more complex classroom activities
    \begin{itemize}
        \item Extend the system beyond single-board, lecture-style classes to settings with multiple focal points (for example, group discussions or multiple boards), so it can handle a wider variety of learning environments.
    \end{itemize}

    \item Enhance real-time analytics
    \begin{itemize}
        \item Improve the live feedback available to instructors by going beyond simple engagement markers, while keeping the analytics lightweight enough to scale across many devices.
    \end{itemize}

    \item Strengthen privacy protections
    \begin{itemize}
        \item Ensure the system does not expose sensitive or personally identifiable information. This includes hiding personal details captured by eye cameras and preventing raw video feeds from being directly viewed by researchers or instructors.
    \end{itemize}
\end{enumerate}

\section{Extras}
These are the extra deliverables were confirmed with the initial meeting with the supervisors of the projects. 
The purpose of these extra deliverables is for the sustainability of the project, making it easy for future developers to build upon. 

\begin{enumerate}
    \item Code Walkthrough Report \\
    This will be a thorough documentation of the files and folders in the GitHub repository associated with the project. 

    \item User Instruction Video \\
    This will be an instructional video to demonstrate how to interact with the dashboard deliverable of the project for each of the user roles (e.g., instructor). 
\end{enumerate}


\newpage{}

\section*{Appendix --- Reflection}

\begin{enumerate}
    \item What went well while writing this deliverable? 
    
\textbf{Stanley:} Writing this deliverable went smoothly for me. The reason for this is that I was responsible for creating the team charter (which is on the development plan document), which was a simple process where I came up with some reasonable responses and checked in to see if my team agreed.

\textbf{Manan:} Writing this deliverable went well because our team was able to communicate effectively and divide the work equally. We were able to discuss our ideas and come to a consensus on how to approach the problem statement and goals. This made the writing process smoother and more efficient.

\textbf{Angela:} After meeting with our supervisors and discussing the plan as a team, we were able to clarify and finalize the details for this deliverable. We also received supporting resources, including research papers and the GitHub repository, which helped us better understand the project requirements and identify the stakeholders it affects.

\textbf{Ann:} During our discussion with our supervisors, a lot of the parts that were initially unclear to me (goals, inputs, outputs) were discussed in detail. Our supervisors provided good resources, previous Github repository links associated with the project, and organized a current document containing details of the project that made the writing process for the problem statement and goals smooth. 


    \item What pain points did you experience during this deliverable, and how
    did you resolve them?

\textbf{Stanley:} The only pain point that I experienced was making sure that everyone agreed with the expectations that I wrote (on the development plan document). One member said that if we should let our group know in advance that we will not be able to attend a meeting, that should be reasonable enough for him/her to not show up, which I updated in the expectations later.  

\textbf{Manan:} One pain point was understanding what exactly the proffersors were looking for in this project. We resolved this by discussing as a team and clarifying any doubts we had with the proffersors incharge. 

\textbf{Angela:} Since our supervisor was assigned closer to the deadline for this deliverable, we needed to accelerate our work and quickly build a strong understanding of the project. We resolved this challenge by holding an in-depth meeting with our supervisors, where we clarified expectations and addressed the sections of the deliverable that were initially unclear.

\textbf{Ann:} Personally, my main pain point was getting a good initial understanding of the project scope with the supervisors. A lot of the information and pre-requisites to the project were new to me as I never worked with eye-tracking devices or participated in any classroom research studies. I had to spend a lot of time dissecting the research paper provided by our supervisors to get good context to understand the problem and goals of the project. 

    \item How did you and your team adjust the scope of your goals to ensure
    they are suitable for a Capstone project (not overly ambitious but also of
    appropriate complexity for a senior design project)?

\textbf{Stanley:} For the most part, we met up with our instructors and team members to discuss how much time and effort we could afford to invest. Through this, we were able to gauge exactly the scope of our project.  

\textbf{Manan:} We adjusted the scope of our goals by discussing as a team and considering the time and resources we had available. We made sure to set realistic and achievable goals that would still challenge us and allow us to learn new skills. We also consulted with our instructors to ensure that our goals were appropriate for a senior design project.

\textbf{Angela:} Initially, we drafted our goals based on assumptions drawn from the potential projects PDF. We later refined these goals by incorporating the rubric, lecture slides, and supervisor feedback, ensuring that they are both realistic in scope and appropriately complex for a Capstone project.

\textbf{Ann:} During our discussion with our supervisors, we initially had many suggestions on the different avenues we wanted to take with the project. However, after further discussion, we ultimately filtered it down our initial goals to match what our supervisors found the most necessary in terms of their needs (visual dashboard, focusing on real-time analytics, etc). By filtering out our initial goals and moving them into our stretch goals, we reduced our problem set for a more achievable and reasonable Capstone project. 

\end{enumerate}  

\end{document}
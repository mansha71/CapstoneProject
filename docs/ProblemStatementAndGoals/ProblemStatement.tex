\documentclass{article}

\usepackage{tabularx}
\usepackage{booktabs}

\title{Problem Statement and Goals\\\progname}

\author{\authname}

\date{}

\input{../Comments}
%% Common Parts

\newcommand{\progname}{Software Engineering} % PUT YOUR PROGRAM NAME HERE
\newcommand{\authname}{Team 21, Visionaries
\\ Angela Zeng
\\ Ann Shi
\\ Ibrahim Sahi
\\ Manan Sharma
\\ Stanley Chen} % AUTHOR NAMES                  

\usepackage{hyperref}
    \hypersetup{colorlinks=true, linkcolor=blue, citecolor=blue, filecolor=blue,
                urlcolor=blue, unicode=false}
    \urlstyle{same}
                                


\begin{document}

\maketitle

\begin{table}[hp]
\caption{Revision History} \label{TblRevisionHistory}
\begin{tabularx}{\textwidth}{llX}
\toprule
\textbf{Date} & \textbf{Developer(s)} & \textbf{Change}\\
\midrule
09/23/2025 & Angela Zeng & Add first draft of problem statement and goals\\
Date2 & Name(s) & Description of changes\\
... & ... & ...\\
\bottomrule
\end{tabularx}
\end{table}

\section{Problem Statement}

Instructors lack access to real-time insights about where students direct their 
attention during learning activities, particularly in large-group settings. Without 
this data, it is difficult to assess engagement, monitor collaboration, and adapt 
teaching strategies to improve the effectiveness of learning. 

There is a need for a system that can capture and analyze group gaze data during 
classroom activities, so that instructors can better understand and respond to 
student attention and engagement in both synchronous and asynchronous 
learning contexts.

\subsection{Problem}

\subsection{Inputs and Outputs}

\wss{Characterize the problem in terms of ``high level'' inputs and outputs.  
Use abstraction so that you can avoid details.}

\subsection{Stakeholders}

\subsection{Environment}

\subsubsection*{Software Environment}

\begin{itemize}
    \item \textbf{Version Control and Collaboration:} GitHub will be used for source code management, issue tracking, project boards, and continuous integration/continuous delivery (CI/CD).
    \item \textbf{Integrated Development Environment (IDE):} Visual Studio Code (VS Code) will serve as the primary IDE.
\end{itemize}

\section{Goals}

\begin{enumerate}
    \item Develop a Learning Platform
    \begin{itemize}
        \item Integrate large-group eye tracking into both synchronous (live classes) 
        and asynchronous (recorded or online activities) learning contexts.
    \end{itemize}

    \item Enable Contextual Data Capture
    \begin{itemize}
        \item Log gaze data alongside classroom learning activities such as passive 
        content viewing (e.g., watching videos) and active group work 
        (e.g., exercises, discussions).
    \end{itemize}

    \item Conduct In-Person Research
    \begin{itemize}
        \item Run a study with instructors and students at McMaster University 
        to evaluate how gaze-based insights affect teaching and learning.
    \end{itemize}

    \item Inform Future System Designs
    \begin{itemize}
        \item Use study findings to guide the development of features like 
        instructor dashboards and real-time gaze analytics.
    \end{itemize}

    \item Tackle Key Technical Challenges
    \begin{itemize}
        \item Address issues in system design, reliable capture of group gaze data, 
        and effective real-time visualization of attention patterns.
    \end{itemize}
\end{enumerate}

\section{Stretch Goals}

\section{Extras}

\wss{For CAS 741: State whether the project is a research project. This
designation, with the approval (or request) of the instructor, can be modified
over the course of the term.}

\wss{For SE Capstone: List your extras.  Potential extras include usability
testing, code walkthroughs, user documentation, formal proof, GenderMag
personas, Design Thinking, etc.  (The full list is on the course outline and in
Lecture 02.) Normally the number of extras will be two.  Approval of the extras
will be part of the discussion with the instructor for approving the project.
The extras, with the approval (or request) of the instructor, can be modified
over the course of the term.}

\newpage{}

\section*{Appendix --- Reflection}

\wss{Not required for CAS 741}

\input{../Reflection.tex}

\begin{enumerate}
    \item What went well while writing this deliverable? 
    \item What pain points did you experience during this deliverable, and how
    did you resolve them?
    \item How did you and your team adjust the scope of your goals to ensure
    they are suitable for a Capstone project (not overly ambitious but also of
    appropriate complexity for a senior design project)?
\end{enumerate}  

\end{document}
\documentclass[12pt, titlepage]{article}

\usepackage{booktabs}
\usepackage{tabularx}
\usepackage{hyperref}
\hypersetup{
    colorlinks,
    citecolor=blue,
    filecolor=black,
    linkcolor=red,
    urlcolor=blue
}
\usepackage[round]{natbib}

%% Comments

\usepackage{color}

\newif\ifcomments\commentstrue %displays comments
%\newif\ifcomments\commentsfalse %so that comments do not display

\ifcomments
\newcommand{\authornote}[3]{\textcolor{#1}{[#3 ---#2]}}
\newcommand{\todo}[1]{\textcolor{red}{[TODO: #1]}}
\else
\newcommand{\authornote}[3]{}
\newcommand{\todo}[1]{}
\fi

\newcommand{\wss}[1]{\authornote{magenta}{SS}{#1}} 
\newcommand{\plt}[1]{\authornote{cyan}{TPLT}{#1}} %For explanation of the template
\newcommand{\an}[1]{\authornote{cyan}{Author}{#1}}

%% Common Parts

\newcommand{\progname}{Software Engineering} % PUT YOUR PROGRAM NAME HERE
\newcommand{\authname}{Team 21, Visionaries
\\ Ahmed Sahi
\\ Angela Zeng
\\ Ann Shi
\\ Manan Sharma
\\ Stanley Chen} % AUTHOR NAMES                  

\usepackage{hyperref}
    \hypersetup{colorlinks=true, linkcolor=blue, citecolor=blue, filecolor=blue,
                urlcolor=blue, unicode=false}
    \urlstyle{same}
                                


\begin{document}

\title{System Verification and Validation Plan for \progname{}} 
\author{\authname}
\date{\today}
	
\maketitle

\pagenumbering{roman}

\section*{Revision History}

\begin{tabularx}{\textwidth}{p{3cm}p{2cm}X}
\toprule {\bf Date} & {\bf Version} & {\bf Notes}\\
\midrule
Date 1 & 1.0 & Notes\\
Date 2 & 1.1 & Notes\\
\bottomrule
\end{tabularx}

~\\
\wss{The intention of the VnV plan is to increase confidence in the software.
However, this does not mean listing every verification and validation technique
that has ever been devised.  The VnV plan should also be a \textbf{feasible}
plan. Execution of the plan should be possible with the time and team available.
If the full plan cannot be completed during the time available, it can either be
modified to ``fake it'', or a better solution is to add a section describing
what work has been completed and what work is still planned for the future.}

\wss{The VnV plan is typically started after the requirements stage, but before
the design stage.  This means that the sections related to unit testing cannot
initially be completed.  The sections will be filled in after the design stage
is complete.  the final version of the VnV plan should have all sections filled
in.}

\newpage

\tableofcontents

\listoftables
\wss{Remove this section if it isn't needed}

\listoffigures
\wss{Remove this section if it isn't needed}

\newpage

\section{Symbols, Abbreviations, and Acronyms}

\renewcommand{\arraystretch}{1.2}
\begin{tabular}{l l} 
  \toprule		
  \textbf{symbol} & \textbf{description}\\
  \midrule 
  T & Test\\
  \bottomrule
\end{tabular}\\

\wss{symbols, abbreviations, or acronyms --- you can simply reference the SRS
  \citep{SRS} tables, if appropriate}

\wss{Remove this section if it isn't needed}

\newpage

\pagenumbering{arabic}

This document ... \wss{provide an introductory blurb and roadmap of the
  Verification and Validation plan}

\section{General Information}

\subsection{Summary}

\wss{Say what software is being tested.  Give its name and a brief overview of
  its general functions.}

\subsection{Objectives}

\wss{State what is intended to be accomplished.  The objective will be around
  the qualities that are most important for your project.  You might have
  something like: ``build confidence in the software correctness,''
  ``demonstrate adequate usability.'' etc.  You won't list all of the qualities,
  just those that are most important.}

\wss{You should also list the objectives that are out of scope.  You don't have 
the resources to do everything, so what will you be leaving out.  For instance, 
if you are not going to verify the quality of usability, state this.  It is also 
worthwhile to justify why the objectives are left out.}

\wss{The objectives are important because they highlight that you are aware of 
limitations in your resources for verification and validation.  You can't do everything, 
so what are you going to prioritize?  As an example, if your system depends on an 
external library, you can explicitly state that you will assume that external library 
has already been verified by its implementation team.}

\subsection{Extras}

\wss{Summarize the extras (if any) that were tackled by this project.  Extras
can include usability testing, code walkthroughs, user documentation, formal
proof, GenderMag personas, Design Thinking, etc.  Extras should have already
been approved by the course instructor as included in your problem statement.
You can use a pull request to update your extras (in TeamComposition.csv or
Repos.csv) if your plan changes as a result of the VnV planning exercise.}

\subsection{Relevant Documentation}

\wss{Reference relevant documentation.  This will definitely include your SRS
  and your other project documents (design documents, like MG, MIS, etc).  You
  can include these even before they are written, since by the time the project
  is done, they will be written.  You can create BibTeX entries for your
  documents and within those entries include a hyperlink to the documents.}

\citet{SRS}

\wss{Don't just list the other documents.  You should explain why they are relevant and 
how they relate to your VnV efforts.}

\section{Plan}

\wss{Introduce this section.  You can provide a roadmap of the sections to
  come.}

\subsection{Verification and Validation Team}

\wss{Your teammates.  Maybe your supervisor.
  You should do more than list names.  You should say what each person's role is
  for the project's verification.  A table is a good way to summarize this information.}

\subsection{SRS Verification}

\wss{List any approaches you intend to use for SRS verification.  This may
  include ad hoc feedback from reviewers, like your classmates (like your
  primary reviewer), or you may plan for something more rigorous/systematic.}

\wss{If you have a supervisor for the project, you shouldn't just say they will
read over the SRS.  You should explain your structured approach to the review.
Will you have a meeting?  What will you present?  What questions will you ask?
Will you give them instructions for a task-based inspection?  Will you use your
issue tracker?}

\wss{Maybe create an SRS checklist?}

\subsection{Design Verification}

\wss{Plans for design verification}

\wss{The review will include reviews by your classmates}

\wss{Create a checklists?}

\subsection{Verification and Validation Plan Verification}

\wss{The verification and validation plan is an artifact that should also be
verified.  Techniques for this include review and mutation testing.}

\wss{The review will include reviews by your classmates}

\wss{Create a checklists?}

\subsection{Implementation Verification}

\wss{You should at least point to the tests listed in this document and the unit
  testing plan.}

\wss{In this section you would also give any details of any plans for static
  verification of the implementation.  Potential techniques include code
  walkthroughs, code inspection, static analyzers, etc.}

\wss{The final class presentation in CAS 741 could be used as a code
walkthrough.  There is also a possibility of using the final presentation (in
CAS741) for a partial usability survey.}

\subsection{Automated Testing and Verification Tools}

\wss{What tools are you using for automated testing.  Likely a unit testing
  framework and maybe a profiling tool, like ValGrind.  Other possible tools
  include a static analyzer, make, continuous integration tools, test coverage
  tools, etc.  Explain your plans for summarizing code coverage metrics.
  Linters are another important class of tools.  For the programming language
  you select, you should look at the available linters.  There may also be tools
  that verify that coding standards have been respected, like flake9 for
  Python.}

\wss{If you have already done this in the development plan, you can point to
that document.}

\wss{The details of this section will likely evolve as you get closer to the
  implementation.}

\subsection{Software Validation}

\wss{If there is any external data that can be used for validation, you should
  point to it here.  If there are no plans for validation, you should state that
  here.}

\wss{You might want to use review sessions with the stakeholder to check that
the requirements document captures the right requirements.  Maybe task based
inspection?}

\wss{For those capstone teams with an external supervisor, the Rev 0 demo should 
be used as an opportunity to validate the requirements.  You should plan on 
demonstrating your project to your supervisor shortly after the scheduled Rev 0 demo.  
The feedback from your supervisor will be very useful for improving your project.}

\wss{For teams without an external supervisor, user testing can serve the same purpose 
as a Rev 0 demo for the supervisor.}

\wss{This section might reference back to the SRS verification section.}

\section{System Tests}

\wss{There should be text between all headings, even if it is just a roadmap of
the contents of the subsections.}

\subsection{Tests for Functional Requirements}

\wss{Subsets of the tests may be in related, so this section is divided into
  different areas.  If there are no identifiable subsets for the tests, this
  level of document structure can be removed.}

\wss{Include a blurb here to explain why the subsections below
  cover the requirements.  References to the SRS would be good here.}

\subsubsection{Area of Testing1}

\wss{It would be nice to have a blurb here to explain why the subsections below
  cover the requirements.  References to the SRS would be good here.  If a section
  covers tests for input constraints, you should reference the data constraints
  table in the SRS.}
		
\paragraph{Title for Test}

\begin{enumerate}

\item{test-id1\\}

Control: Manual versus Automatic
					
Initial State: 
					
Input: 
					
Output: \wss{The expected result for the given inputs.  Output is not how you
are going to return the results of the test.  The output is the expected
result.}

Test Case Derivation: \wss{Justify the expected value given in the Output field}
					
How test will be performed: 
					
\item{test-id2\\}

Control: Manual versus Automatic
					
Initial State: 
					
Input: 
					
Output: \wss{The expected result for the given inputs}

Test Case Derivation: \wss{Justify the expected value given in the Output field}

How test will be performed: 

\end{enumerate}

\subsubsection{Area of Testing2}

...

\subsection{Tests for Nonfunctional Requirements}

\wss{The nonfunctional requirements for accuracy will likely just reference the
  appropriate functional tests from above.  The test cases should mention
  reporting the relative error for these tests.  Not all projects will
  necessarily have nonfunctional requirements related to accuracy.}

\wss{For some nonfunctional tests, you won't be setting a target threshold for
passing the test, but rather describing the experiment you will do to measure
the quality for different inputs.  For instance, you could measure speed versus
the problem size.  The output of the test isn't pass/fail, but rather a summary
table or graph.}

\wss{Tests related to usability could include conducting a usability test and
  survey.  The survey will be in the Appendix.}

\wss{Static tests, review, inspections, and walkthroughs, will not follow the
format for the tests given below.}

\wss{If you introduce static tests in your plan, you need to provide details.
How will they be done?  In cases like code (or document) walkthroughs, who will
be involved? Be specific.}

\subsubsection{Area of Testing1}
		
\paragraph{Title for Test}

\begin{enumerate}

\item{TC-NFR-1-Latency\\} 
Type: Dynamic, Automated\\ 

Initial State: Both backend and frontend are running locally and the system is connected to mock eye tracking data source.\\

Input/Condition: Conduct a 30 min, test session with simulated eye tracking and video data. Then record the timestamps of when data is recorded, processed and displayed.\\ 

Output/Result: Record the average latency between capturing and visualizing data. Pass if latency is less than 1.5 s.\\ 

How test will be performed: Log the timestamp when data is inputted, and when the dashboard is updated. Then compute the delay from the logs of timestamps.\\ 

\item{TC-NFR-2-UpdateHz\\} 
Type: Dynamic, Automated\\ 

Initial State: Dashboard is rendering heatmaps in browser.\\ 

Input/Condition: For a 10 min session, with browser performance logs recording the render times.\\ 

Output/Result: Average and 95th percentile render frequency. Test will pass if the mean is greater than 20 Hz and the 95th percentile is greater than 18 Hz. This will include FPS histogram and timeline graph.\\ 

How test will be performed: The performance at each render will be recorded by using a Javascript probe, then the data will be summarized automatically in the command line.\\ 

\item{TC-NFR-3-SyncDrift\\}
Type: Dynamic, Automated\\

Initial State: 3 to 10 devices synchronized via Network Time Protocol (NTP), all streaming gaze and egoview data under standard classroom conditions.\\

Input/Condition: Each device emits synchronization markers once per second during a session.\\

Output/Result: Time-difference series between devices with maximum drift per minute recorded. Pass if drift stays within 20 ms.\\

How test will be performed: Match sync markers by frame index across devices and compute drift statistics to verify requirement.

\item{TC-NFR-4-JitterLoss\\}
Type: Dynamic, Automated\\

Initial State: Live data streaming between devices and analytics backend.\\

Input/Condition: Browser script sends WebSocket heartbeat every 100 ms while measuring round-trip delay.\\

Output/Result: Logs containing average, P95, and maximum jitter and a chart of packet loss vs jitter.\\

How test will be performed: Browser records timestamps for sent/received messages, computes latency variation, and summarizes results in dashboard or CI logs.

\item{TC-NFR-16-Scale10\\}
Type: Load\\

Initial State: Backend services running, synthetic gaze stream generators configured.\\

Input/Condition: Simulate 1, 3, 5, and 10 devices concurrently for up to 20 minutes each, streaming gaze data at normal rates.\\

Output/Result: CPU and memory usage, latency percentiles, and sync error recorded. Pass if sync error < 20 ms at 10 devices. Produce plots of latency vs device count and sync error vs device count.\\

How test will be performed: Use synthetic streams to simulate multiple devices while collecting system metrics; visualize results.

\item{TC-NFR-5-AutoRecover\\}
Type: Dynamic, Automated\\

Initial State: Multiple devices actively streaming gaze data to backend.\\

Input/Condition: Intentionally disconnect each device stream for 10 seconds at randomized intervals, repeated 20 times.\\

Output/Result: Reconnection time for each disconnection; pass if all streams automatically reconnect without data loss or corruption.\\

How test will be performed: Scripted disconnections; backend verifies reconnection by checking frame sequence continuity and comparing checksums before/after reconnection.

\item{TC-NFR-6-Uptime\\}
Type: Long-run observation\\

Initial State: Full classroom analytics system deployed and running under normal conditions.\\

Input/Condition: Conduct three scheduled 90-minute sessions across one week while tracking uptime metrics.\\

Output/Result: Summary report with total uptime percentage, number of downtime incidents, and average recovery time.\\

How test will be performed: Send a signal every 10 seconds; missed signals logged as downtime and recovery duration measured.

\item{TC-NFR-17-TimeACID\\}
Type: Dynamic, Static\\

Initial State: Database running and handling necessary operations.\\

Input/Condition: Insert 10,000 events with known timestamps while performing multiple transactions.\\

Output/Result: Timestamp drift within 33 ms and ACID properties maintained.\\

How test will be performed: Compare database timestamps to ingestion clock and verify transactions commit/rollback correctly using transactional test suite.

\item{TC-NFR-18-BigSession\\}
Type: Load test\\

Initial State: System ready for large-scale data intake.\\

Input/Condition: Process 50 x 200 GB of generated session data while running ingestion and analytics queries concurrently.\\

Output/Result: Throughput and query latency remain stable; record any performance degradation or curves.\\

How test will be performed: Batch load generated data, monitor CPU/memory (Prometheus), and measure query response times under increasing load.

\item{TC-NFR-26-BackupRestore\\}
Type: Static, Dynamic\\

Initial State: Backup configuration active and scheduled.\\

Input/Condition: Trigger full system backup and restore into clean database instance.\\

Output/Result: Verify encryption during backup and exact match of restored data; pass if checksums and verification logs match.\\

How test will be performed: Activate backup workflow, restore data, and perform row-by-row hash comparison to confirm data integrity.

\subsubsection{Usability and Accessibility}

\item{TC-NFR-7-Onboarding\\}
Type: Usability study, task-based inspection\\

Initial State: Dashboard deployed and accessible to first-time users.\\

Input/Condition: New instructors complete key tasks (start session, view heatmaps, export data).\\

Output/Result: Users complete tasks within expected onboarding time; collect ease-of-use notes and feedback.\\

How test will be performed: Observe task completion times and collect short post-task surveys.

\item{TC-NFR-8-Contrast\\}
Type: Static, Dynamic\\

Initial State: Dashboard functional in web browser.\\

Input/Condition: Run automated accessibility audits and manual checks under various lighting.\\

Output/Result: UI elements meet WCAG AA contrast standards and are legible under classroom lighting.\\

How test will be performed: Use Lighthouse or axe and verify results visually with contrast checkers.

\item{TC-NFR-21-UCD\\}
Type: Iterative usability study\\

Initial State: Dashboard prototype ready for user testing.\\

Input/Condition: Conduct multiple usability sessions with instructors and TAs using realistic scenarios.\\

Output/Result: Improvements in task completion rates, reduced errors, and higher satisfaction across iterations.\\

How test will be performed: Run structured sessions, analyze feedback, and compare results across iterations.

\subsubsection{Security and Privacy}

\item{TC-NFR-9-TLSOnly\\}

Type: Static, Dynamic\\

Initial State: Application deployed with HTTPS proxy enabled.\\

Input/Condition: Attempt requests using both HTTP and HTTPS.\\

Output/Result: HTTP requests blocked or redirected; HTTPS succeeds with valid certificate.\\

How test will be performed: Use curl and browser devtools to confirm redirects and certificate validity.

\item{TC-NFR-10-RBAC\\}
Type: Dynamic\\

Initial State: Backend roles configured for instructor and researcher accounts.\\

Input/Condition: Send API requests under different roles to verify access control.\\

Output/Result: Only authorized roles access protected endpoints; unauthorized requests rejected.\\

How test will be performed: Use Postman or automated scripts with varied tokens to test role enforcement.

\item{TC-NFR-11-AnonStore\\}
Type: Static, Dynamic\\

Initial State: Database populated with session data.\\

Input/Condition: Inspect stored session records and IDs.\\

Output/Result: No personal identifiers stored; only pseudonyms or session IDs used.\\

How test will be performed: Run SQL queries and manually review samples to ensure anonymization.

\item{TC-NFR-24-Consent\\}
Type: Static, Dynamic\\

Initial State: Application ready for a new recording session.\\

Input/Condition: Attempt to start recording without completing consent process.\\

Output/Result: Recording cannot start until consent explicitly confirmed.\\

How test will be performed: Observe UI behavior and check consent status recorded in database.

\subsubsection{Portability, Maintainability, and Process}

\item{TC-NFR-12-CrossPlat\\}
Type: Dynamic\\

Initial State: System containerized and ready for deployment.\\

Input/Condition: Run setup on Windows, macOS, and Ubuntu.\\

Output/Result: Application runs successfully with no critical functional issues on all platforms.\\

How test will be performed: Use Docker Compose or equivalent to deploy on each OS and verify key flows.

\item{TC-NFR-13-Linters\\}
Type: Static\\

Initial State: Source code repository available.\\

Input/Condition: Run ESLint for JS/TS and flake8/ruff for Python.\\

Output/Result: No linting errors or style violations reported.\\

How test will be performed: Run linting tools in CI and fix or document any findings.

\item{TC-NFR-14-CI\\}
Type: Process verification

Initial State: Continuous integration (CI) pipeline configured.

Input/Condition: Push a new pull request to trigger CI.

Output/Result: The build, lint, and test stages run automatically and pass without errors.

How test will be performed: Review CI logs and confirm successful artifact generation.

\item{TC-NFR-15-Config\\}
Type: Static, dynamic

Initial State: Configuration files and environment variables defined.

Input/Condition: Modify configuration values without changing the code.

Output/Result: Application updates behavior correctly based on new configurations.

How test will be performed: Edit .env values or config files and restart the service to verify changes take effect.

\item{TC-NFR-19-License\\}
Type: Static

Initial State: Dependency list available.

Input/Condition: Run a license scan across all dependencies.

Output/Result: All dependencies comply with NCRL-1.0 or compatible licenses.

How test will be performed: Use tools like license-checker to scan and validate license information.

\item{TC-NFR-20-NonComm\\}
Type: Static

Initial State: Documentation and UI ready for review.

Input/Condition: Review license text and all visible legal disclaimers.

Output/Result: Academic use only or equivalent notice is clearly displayed in the documentation and UI.

How test will be performed: Manually inspect the README, EULA, and app splash screen.

\item{TC-NFR-23-EnvProtocol\\}
Type: Checklist, observation

Initial State: Classroom setup complete.

Input/Condition: Run the environmental calibration checklist.

Output/Result: All requirements lighting, distance, calibration, and setup are met and documented.

How test will be performed: Complete the checklist, take supporting photos, and attach them to the report.

\item{TC-NFR-25-Comfort\\}
Type: Observation, survey

Initial State: Session underway with active participants.

Input/Condition: Gather short post-session feedback from participants.

Output/Result: Record comfort ratings using a Likert scale and collect optional comments.

How test will be performed: Conduct an anonymous feedback survey followed by a brief debrief session.

\subsubsection{Research Validity}

\item{TC-NFR-22-Model\\}
Type: Dynamic, review

Initial State: Analytics model implemented and trained.

Input/Condition: Run the model on a labeled dataset with known ground truth.

Output/Result: Model predictions closely match reference results; report scalar relative error and vector norm differences.

How test will be performed: Compare model outputs with the reference dataset, calculate error metrics, and document findings, including any observed bias.

\end{enumerate}

...

\subsection{Traceability Between Test Cases and Requirements}

\begin{table}[htbp]
\centering
\caption{Traceability Between NFRs and Test Cases}
\renewcommand{\arraystretch}{1.1}
\small
\begin{tabular}{p{2cm} p{4cm} p{7cm}}
\hline
\textbf{NFR (SRS)} & \textbf{Test Case ID(s)} & \textbf{Area} \\
\hline
NFR-1 & TC-NFR-1-Latency & A1: Data Pipeline \& Ingestion \\
NFR-2 & TC-NFR-2-UpdateHz & A2: Visualization \& Rendering \\
NFR-3 & TC-NFR-3-SyncDrift, TC-NFR-16-Scale10 & A1: Multi-Device Synchronization \\
NFR-4 & TC-NFR-4-JitterLoss & A1: Network Stability \\
NFR-5 & TC-NFR-5-AutoRecover & A1: Data Pipeline Reliability \\
NFR-6 & TC-NFR-6-Uptime & A5: Operational Resilience \\
NFR-7 & TC-NFR-7-Onboarding & A4: Usability \& UI Flow \\
NFR-8 & TC-NFR-8-Contrast & A4: Accessibility \\
NFR-9 & TC-NFR-9-TLSOnly & A3: Security (Transport) \\
NFR-10 & TC-NFR-10-RBAC & A3: Security (Access Control) \\
NFR-11 & TC-NFR-11-AnonStore & A3: Privacy \& Storage \\
NFR-12 & TC-NFR-12-CrossPlat & A5: Portability \\
NFR-13 & TC-NFR-13-Linters & A5: Maintainability (Code Quality) \\
NFR-14 & TC-NFR-14-CI & A5: Maintainability (CI/CD) \\
NFR-15 & TC-NFR-15-Config & A5: Maintainability (Configuration) \\
NFR-16 & TC-NFR-16-Scale10, TC-NFR-3-SyncDrift & A1: Scalability \& Multi-Device Load \\
NFR-17 & TC-NFR-17-TimeACID & A3: Database Integrity \\
NFR-18 & TC-NFR-18-BigSession & A3: Backend Performance \\
NFR-19 & TC-NFR-19-License & A5: Compliance \\
NFR-20 & TC-NFR-20-NonComm & A5: Compliance \\
NFR-21 & TC-NFR-21-UCD & A4: Usability Iteration \\
NFR-22 & TC-NFR-22-Model & A2: Analytics \& Validation \\
NFR-23 & TC-NFR-23-EnvProtocol & A1: Environmental Setup \\
NFR-24 & TC-NFR-24-Consent & A3: Privacy \& Consent Flow \\
NFR-25 & TC-NFR-25-Comfort & A4: User Study Feedback \\
NFR-26 & TC-NFR-26-BackupRestore & A3: Data Backup \& Recovery \\
\hline
\end{tabular}
\end{table}

\noindent
\textit{Note.} The “Area” column groups related quality attributes:
\begin{itemize}
  \item \textbf{A1:} Data Pipeline, Ingestion, and Device Sync  
  \item \textbf{A2:} Visualization and Analytics  
  \item \textbf{A3:} Data Management, Security, and Privacy  
  \item \textbf{A4:} Usability and User Experience  
  \item \textbf{A5:} Maintainability, Portability, and Compliance  
\end{itemize}


\section{Unit Test Description}

\wss{This section should not be filled in until after the MIS (detailed design
  document) has been completed.}

\wss{Reference your MIS (detailed design document) and explain your overall
philosophy for test case selection.}  

\wss{To save space and time, it may be an option to provide less detail in this section.  
For the unit tests you can potentially layout your testing strategy here.  That is, you 
can explain how tests will be selected for each module.  For instance, your test building 
approach could be test cases for each access program, including one test for normal behaviour 
and as many tests as needed for edge cases.  Rather than create the details of the input 
and output here, you could point to the unit testing code.  For this to work, you code 
needs to be well-documented, with meaningful names for all of the tests.}

\subsection{Unit Testing Scope}

\wss{What modules are outside of the scope.  If there are modules that are
  developed by someone else, then you would say here if you aren't planning on
  verifying them.  There may also be modules that are part of your software, but
  have a lower priority for verification than others.  If this is the case,
  explain your rationale for the ranking of module importance.}

\subsection{Tests for Functional Requirements}

\wss{Most of the verification will be through automated unit testing.  If
  appropriate specific modules can be verified by a non-testing based
  technique.  That can also be documented in this section.}

\subsubsection{Module 1}

\wss{Include a blurb here to explain why the subsections below cover the module.
  References to the MIS would be good.  You will want tests from a black box
  perspective and from a white box perspective.  Explain to the reader how the
  tests were selected.}

\begin{enumerate}

\item{test-id1\\}

Type: \wss{Functional, Dynamic, Manual, Automatic, Static etc. Most will
  be automatic}
					
Initial State: 
					
Input: 
					
Output: \wss{The expected result for the given inputs}

Test Case Derivation: \wss{Justify the expected value given in the Output field}

How test will be performed: 
					
\item{test-id2\\}

Type: \wss{Functional, Dynamic, Manual, Automatic, Static etc. Most will
  be automatic}
					
Initial State: 
					
Input: 
					
Output: \wss{The expected result for the given inputs}

Test Case Derivation: \wss{Justify the expected value given in the Output field}

How test will be performed: 

\item{...\\}
    
\end{enumerate}

\subsubsection{Module 2}

...

\subsection{Tests for Nonfunctional Requirements}

\wss{If there is a module that needs to be independently assessed for
  performance, those test cases can go here.  In some projects, planning for
  nonfunctional tests of units will not be that relevant.}

\wss{These tests may involve collecting performance data from previously
  mentioned functional tests.}

\subsubsection{Module ?}
		
\begin{enumerate}

\item{test-id1\\}

Type: \wss{Functional, Dynamic, Manual, Automatic, Static etc. Most will
  be automatic}
					
Initial State: 
					
Input/Condition: 
					
Output/Result: 
					
How test will be performed: 
					
\item{test-id2\\}

Type: Functional, Dynamic, Manual, Static etc.
					
Initial State: 
					
Input: 
					
Output: 
					
How test will be performed: 

\end{enumerate}

\subsubsection{Module ?}

...

\subsection{Traceability Between Test Cases and Modules}

\wss{Provide evidence that all of the modules have been considered.}
				
\bibliographystyle{plainnat}

\bibliography{../../refs/References}

\newpage

\section{Appendix}

This is where you can place additional information.

\subsection{Symbolic Parameters}

The definition of the test cases will call for SYMBOLIC\_CONSTANTS.
Their values are defined in this section for easy maintenance.

\subsection{Usability Survey Questions?}

\wss{This is a section that would be appropriate for some projects.}

\newpage{}
\section*{Appendix --- Reflection}

\wss{This section is not required for CAS 741}

The information in this section will be used to evaluate the team members on the
graduate attribute of Lifelong Learning.

The purpose of reflection questions is to give you a chance to assess your own
learning and that of your group as a whole, and to find ways to improve in the
future. Reflection is an important part of the learning process.  Reflection is
also an essential component of a successful software development process.  

Reflections are most interesting and useful when they're honest, even if the
stories they tell are imperfect. You will be marked based on your depth of
thought and analysis, and not based on the content of the reflections
themselves. Thus, for full marks we encourage you to answer openly and honestly
and to avoid simply writing ``what you think the evaluator wants to hear.''

Please answer the following questions.  Some questions can be answered on the
team level, but where appropriate, each team member should write their own
response:


\begin{enumerate}
  \item What went well while writing this deliverable? 

  Manan: The team collaborated effectively, dividing tasks based on individual strengths, which streamlined the process. We also held regular meetings to discuss progress and address any challenges promptly.
  \item What pain points did you experience during this deliverable, and how
    did you resolve them?

  Manan: One challenge was ensuring consistency in formatting and terminology across sections written by different team members. We resolved this by designating a final editor who reviewed the entire document for coherence and uniformity.
  \item What knowledge and skills will the team collectively need to acquire to
  successfully complete the verification and validation of your project?
  Examples of possible knowledge and skills include dynamic testing knowledge,
  static testing knowledge, specific tool usage, Valgrind etc.  You should look to
  identify at least one item for each team member.

  Manan: The team will need to acquire knowledge in dynamic testing methodologies, static code analysis techniques, and proficiency in using automated testing tools such as pytest for Python and Jest for JavaScript.
  \item For each of the knowledge areas and skills identified in the previous
  question, what are at least two approaches to acquiring the knowledge or
  mastering the skill?  Of the identified approaches, which will each team
  member pursue, and why did they make this choice?

  Manan: To acquire knowledge in dynamic testing methodologies, team members can attend workshops or webinars focused on software testing best practices, and also engage in hands-on practice by writing and executing test cases for our project.
\end{enumerate}

\end{document}
\documentclass{article}

\usepackage{float}
\restylefloat{table}

\usepackage{booktabs}

\title{Team Contributions: POC\\\progname}

\author{\authname}

\date{}

\input{../Comments}
%% Common Parts

\newcommand{\progname}{Software Engineering} % PUT YOUR PROGRAM NAME HERE
\newcommand{\authname}{Team 21, Visionaries
\\ Angela Zeng
\\ Ann Shi
\\ Ibrahim Sahi
\\ Manan Sharma
\\ Stanley Chen} % AUTHOR NAMES                  

\usepackage{hyperref}
    \hypersetup{colorlinks=true, linkcolor=blue, citecolor=blue, filecolor=blue,
                urlcolor=blue, unicode=false}
    \urlstyle{same}
                                


\begin{document}

\maketitle

This document summarizes the contributions of each team member up to the POC
Demo.  The time period of interest is the time between the beginning of the term
and the POC demo.

\section{Demo Plans}

\wss{What will you be demonstrating}

\section{Team Meeting Attendance}

\wss{For each team member how many team meetings have they attended over the
time period of interest.  This number should be determined from the meeting
issues in the team's repo.  The first entry in the table should be the total
number of team meetings held by the team.}

\begin{table}[H]
\centering
\begin{tabular}{ll}
\toprule
\textbf{Student} & \textbf{Meetings}\\
\midrule
Total & Num\\
Name 1 & Num\\
Name 2 & Num\\
Name 3 & Num\\
Name 4 & Num\\
Name 5 & Num\\
\bottomrule
\end{tabular}
\end{table}

\wss{If needed, an explanation for the counts can be provided here.}

\section{Supervisor/Stakeholder Meeting Attendance}

\wss{For each team member how many supervisor/stakeholder team meetings have
they attended over the time period of interest.  This number should be determined
from the supervisor meeting issues in the team's repo.  The first entry in the
table should be the total number of supervisor and team meetings held by the
team.  If there is no supervisor, there will usually be meetings with
stakeholders (potential users) that can serve a similar purpose.}

\noindent \textbf{Supervisor's Name: } Dr. Irene Yuan, Dr. Lauren Fink

\begin{table}[H]
\centering
\begin{tabular}{ll}
\toprule
\textbf{Student} & \textbf{Meetings}\\
\midrule
Total & 5\\
Angela Zeng & 4\\
Ann Shi & 5\\
Ibrahim Sahi & 4\\
Manan Sharma & 4\\
Stanley Chen & 4\\
\bottomrule
\end{tabular}
\end{table}

\section{Lecture Attendance}

\begin{table}[H]
\centering
\begin{tabular}{ll}
\toprule
\textbf{Student} & \textbf{Lectures}\\
\midrule
Total & 13\\
Angela Zeng & 7\\
Ann Shi & 6\\
Ibrahim Sahi & 6\\
Manan Sharma & 5\\
Stanley Chen & 6\\
\bottomrule
\end{tabular}
\end{table}

\section{TA Document Discussion Attendance}

\noindent \textbf{TA's Name: } [Lucas Dutton]

\begin{table}[H]
\centering
\begin{tabular}{ll}
\toprule
\textbf{Student} & \textbf{Lectures}\\
\midrule
Total & 3\\
Angela Zeng & 3\\
Ann Shi & 2\\
Manan Sharma & 3\\
Ibrahim Sahi & 2\\
Stanley Chen & 1\\
\bottomrule
\end{tabular}
\end{table}

\section{Commits}

\begin{table}[H]
\centering
\begin{tabular}{lll}
\toprule
\textbf{Student} & \textbf{Commits} & \textbf{Percent}\\
\midrule
Total & 72 & 100\% \\
Angela Zeng & 18 & 25\% \\
Ann Shi & 17 & 24\% \\
Manan Sharma & 18 & 25\% \\
Ibrahim Sahi & 12 & 17\% \\
Stanley Chen & 7 & 10\% \\
\bottomrule
\end{tabular}
\end{table}

\section{Issue Tracker}

\wss{For each team member how many issues have they authored (including open and
closed issues (O+C)) and how many have they been assigned (only counting closed
issues (C only)) over the time period of interest.}

\begin{table}[H]
\centering
\begin{tabular}{lll}
\toprule
\textbf{Student} & \textbf{Authored (O+C)} & \textbf{Assigned (C only)}\\
\midrule
Name 1 & Num & Num \\
Name 2 & Num & Num \\
Name 3 & Num & Num \\
Name 4 & Num & Num \\
Name 5 & Num & Num \\
\bottomrule
\end{tabular}
\end{table}

\wss{If needed, an explanation for the counts can be provided here.}

\section{CICD}

\wss{Say how CICD will be used in your project}

\section{Team Charter Trigger Items}

\wss{Provide a summary of the quantified triggers identified in the team's
charter.}

\wss{Provide a list of any violations of the triggers.  If the team wishes, the
violations can be summarized on aggregate, instead of naming specific team
members.}

\wss{Provide a plan to address the violations.  This could include revising the
triggers, if they are found to be too weak, strong or ambiguous.}

\section{Additional Productivity Metrics}

\wss{If your team has additional metrics of productivity, please feel free to
add them to this report.}

\end{document}

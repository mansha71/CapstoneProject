\documentclass{article}

\usepackage{float}
\restylefloat{table}

\usepackage{booktabs}

\title{Team Productivity: Rev 0\\\progname}

\author{\authname}

\date{}

\input{../Comments.tex}
%% Common Parts

\newcommand{\progname}{Software Engineering} % PUT YOUR PROGRAM NAME HERE
\newcommand{\authname}{Team 21, Visionaries
\\ Angela Zeng
\\ Ann Shi
\\ Ibrahim Sahi
\\ Manan Sharma
\\ Stanley Chen} % AUTHOR NAMES                  

\usepackage{hyperref}
    \hypersetup{colorlinks=true, linkcolor=blue, citecolor=blue, filecolor=blue,
                urlcolor=blue, unicode=false}
    \urlstyle{same}
                                


\begin{document}

\maketitle

This document summarizes the contributions of each team member for the Rev 0
Demo.  The time period of interest is the time between the PoC demo and the Rev
0 demo; the contributions prior to the PoC are NOT included.

\section{Demo Plans}

\wss{What will you be demonstrating}

\section{Team Meeting Attendance}

\begin{table}[H]
\centering
\begin{tabular}{ll}
\toprule
\textbf{Student} & \textbf{Meetings}\\
\midrule
Total & 5\\
Angela Zeng & 4\\
Ann Shi & 4\\
Manan Sharma & 4\\
Ibrahim Sahi & 4\\
Stanley Chen & 4\\
\bottomrule
\end{tabular}
\end{table}

\section{Supervisor/Stakeholder Meeting Attendance}

\wss{For each team member how many supervisor/stakeholder team meetings have
they attended over the time period of interest.  This number should be determined
from the supervisor meeting issues in the team's repo.  The first entry in the
table should be the total number of supervisor and team meetings held by the
team.  If there is no supervisor, there will usually be meetings with
stakeholders (potential users) that can serve a similar purpose.}

\noindent \textbf{Supervisor's Name: Lauren Fink} 

\begin{table}[H]
\centering
\begin{tabular}{ll}
\toprule
\textbf{Student} & \textbf{Meetings}\\
\midrule
Total & Num\\
Angela Zeng & Num\\
Ann Shi & Num\\
Manan Sharma & Num\\
Ibrahim Sahi & Num\\
Stanley Chen & Num\\
\bottomrule
\end{tabular}
\end{table}

\wss{If needed, an explanation for the counts can be provided here.}

\section{Lecture Attendance}

\begin{table}[H]
\centering
\begin{tabular}{ll}
\toprule
\textbf{Student} & \textbf{Lectures}\\
\midrule
Total & Num\\
Angela Zeng & 1\\
Ann Shi & 0\\
Manan Sharma & 1\\
Ibrahim Sahi & 0\\
Stanley Chen & 0\\
\bottomrule
\end{tabular}
\end{table}

\section{TA Document Discussion Attendance}

\noindent \textbf{TA's Name: Lucas}

\begin{table}[H]
\centering
\begin{tabular}{ll}
\toprule
\textbf{Student} & \textbf{Meetings}\\
\midrule
Total & Num\\
Angela Zeng & 0\\
Ann Shi & 0\\
Manan Sharma & 0\\
Ibrahim Sahi & 0\\
Stanley Chen & 0\\
\bottomrule
\end{tabular}
\end{table}

\section{Commits}

\begin{table}[H]
\centering
\begin{tabular}{lllll}
\toprule
\textbf{Student} & \textbf{Commits} & \textbf{Percent} & \textbf{Lines Added} &
\textbf{Lines Deleted}\\
\midrule
Total & 204 & 100\% & 26328 & 7228\\
Angela Zeng & 64 & 31\% & 7390 & 1367\\
Ann Shi & 53 & 26\% & 2055 & 1683\\
Manan Sharma & 50 & 25\% & 12805 & 575\\
Ibrahim Sahi & 32 & 16\% & 1027 & 709\\
Stanley Chen & 5 & 2\% & 0 & 0\\
\bottomrule
\end{tabular}
\end{table}

The number of commits made may not accurately depict the amount of work that has been done by each member. Work was done by Manan, Ann and Stanley on the realtime application of the project on a seperate unmerged branch of a seperate private repository (SocialEyes) made from our supervisors. 13 commits were made with a total of 208 additions and 56 deletions.
Angela and Ibrahim worked on more of the research side of the project, gathering data and information of various computer vision libraries to help with the development of the project.


\section{Issue Tracker}

\begin{table}[H]
\centering
\begin{tabular}{lll}
\toprule
\textbf{Student} & \textbf{Authored (O+C)} & \textbf{Assigned (C only)}\\
\midrule
Angela Zeng & 41 & 15 \\
Ann Shi & 27 & 24 \\
Manan Sharma & 10 & 6 \\
Ibrahim Sahi & 2 & 1 \\
Stanley Chen & 8 & 6 \\
\bottomrule
\end{tabular}
\end{table}

\section{CICD via GitHub Actions}

\wss{Say how CICD technology is used and will be used in your project.}

\wss{Provide links to your CICD yaml files}

\section{Extras}

\wss{What is the plan (as documented in TeamComposition.csv) for the team's
extras?  Should the extras be modified now that the team knows more about the
project?}


\section{Team Charter Trigger Items}

The team established clear triggers to maintain accountability and progress. Members are expected to attend one tutorial meeting and one additional weekly meeting, with a 10-minute grace period for lateness. Missing meetings without prior notice or a valid excuse is considered a violation. Tasks are expected to be completed by agreed-upon deadlines, and consistently late or incomplete work triggers follow-up and workload redistribution. Repeated lack of participation, poor-quality work, or uncooperative behavior are also treated as trigger conditions that require team discussion and, if needed, escalation to course staff.

Here are some of the violations that were noted: 
\begin{itemize}
    \item Minor punctuality issues occurred occasionally, with some team members arriving late to meetings.
    \item A small number of tasks were submitted past their deadlines due to scheduling or workload constraints.
\end{itemize}

To address these issues, the team created a dedicated Discord channel to clearly communicate deadlines and emphasize timely submissions. For punctuality, we agreed that occasional violations of the 10-minute grace period are acceptable, provided they do not become a recurring pattern.

\section{Additional Productivity Metrics}

\wss{If your team has additional metrics of productivity, please feel free to
add them to this report.  If not, please explicitly state that there are no additional
metrics.}

\wss{Additional metrics can include things like code reviews done, pull requests
created, count of joining meetings late, count of number of times contributions
had to be corrected, number of internal deadlines missed, test cases written, etc.}

\wss{We are looking for data on these metrics, not just a list of additional
metrics the team is planning on tracking.  However, if all you have is a plan,
please share it here.}

\end{document}
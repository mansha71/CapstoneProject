\documentclass[12pt, titlepage]{article}

\usepackage{booktabs}
\usepackage{tabularx}
\usepackage{hyperref}
\hypersetup{
    colorlinks,
    citecolor=blue,
    filecolor=black,
    linkcolor=red,
    urlcolor=blue
}
\usepackage[round]{natbib}

%% Comments

\usepackage{color}

\newif\ifcomments\commentstrue %displays comments
%\newif\ifcomments\commentsfalse %so that comments do not display

\ifcomments
\newcommand{\authornote}[3]{\textcolor{#1}{[#3 ---#2]}}
\newcommand{\todo}[1]{\textcolor{red}{[TODO: #1]}}
\else
\newcommand{\authornote}[3]{}
\newcommand{\todo}[1]{}
\fi

\newcommand{\wss}[1]{\authornote{magenta}{SS}{#1}} 
\newcommand{\plt}[1]{\authornote{cyan}{TPLT}{#1}} %For explanation of the template
\newcommand{\an}[1]{\authornote{cyan}{Author}{#1}}

%% Common Parts

\newcommand{\progname}{Software Engineering} % PUT YOUR PROGRAM NAME HERE
\newcommand{\authname}{Team 21, Visionaries
\\ Ahmed Sahi
\\ Angela Zeng
\\ Ann Shi
\\ Manan Sharma
\\ Stanley Chen} % AUTHOR NAMES                  

\usepackage{hyperref}
    \hypersetup{colorlinks=true, linkcolor=blue, citecolor=blue, filecolor=blue,
                urlcolor=blue, unicode=false}
    \urlstyle{same}
                                


\begin{document}

\title{System Verification and Validation Plan for \progname{}} 
\author{\authname}
\date{\today}
	
\maketitle

\pagenumbering{roman}

\section*{Revision History}

\begin{tabularx}{\textwidth}{p{3cm}p{2cm}X}
\toprule {\bf Date} & {\bf Version} & {\bf Notes}\\
\midrule
Date 1 & 1.0 & Notes\\
Date 2 & 1.1 & Notes\\
\bottomrule
\end{tabularx}

~\\
\wss{The intention of the VnV plan is to increase confidence in the software.
However, this does not mean listing every verification and validation technique
that has ever been devised.  The VnV plan should also be a \textbf{feasible}
plan. Execution of the plan should be possible with the time and team available.
If the full plan cannot be completed during the time available, it can either be
modified to ``fake it'', or a better solution is to add a section describing
what work has been completed and what work is still planned for the future.}

\wss{The VnV plan is typically started after the requirements stage, but before
the design stage.  This means that the sections related to unit testing cannot
initially be completed.  The sections will be filled in after the design stage
is complete.  the final version of the VnV plan should have all sections filled
in.}

\newpage

\tableofcontents

\newpage

\section{Symbols, Abbreviations, and Acronyms}

\renewcommand{\arraystretch}{1.2}
\begin{tabular}{l p{10cm}} 
  \toprule		
  \textbf{Acronym} & \textbf{Description}\\
  \midrule 
  ET & Eye-tracking: the process of measuring where a person is looking (the gaze point) using specialized hardware, such as wearable glasses.\\
  NTP & Network Time Protocol: used to align timestamps across multiple devices, ensuring synchronized data streams.\\
  CI/CD & Continuous Integration / Continuous Delivery: a software engineering practice involving automated building, testing, and deployment pipelines.\\
  API & Application Programming Interface: defined methods and data formats that allow system components or external applications to communicate.\\
  RBAC & Role-based Access Control: a security model that restricts system access based on user roles and permissions.\\
  POC & Proof-of-Concept: the initial implementation phase (Rev 0) focusing on single-device operation, local data processing, and dashboard visualization.\\
  \bottomrule
\end{tabular}


\newpage

\pagenumbering{arabic}

This document outlines the Verification and Validation (VnV) plan for the group eye-tracking learning platform developed in this capstone project. The purpose of this plan is to establish a structured approach for verifying that the system has been built according to its specifications and validating that it meets its intended purpose.
The plan will detail the methods, tools, and responsibilities involved in increasing the confidence of the developed software. It includes documentation of general information, team roles, verification procedures across the system’s lifecycle stages (requirements, design, and implementation), and both system plus the eventual addition of unit-level test specifications. The document also defines the testing framework for functional and non-functional requirements, along with traceability between requirements, modules, and test cases.

\section{General Information}

\subsection{Summary}

The software being tested is an enhanced version of the \textbf{SocialEyes} framework, a system originally designed for recording and analyzing group gaze data. This project extends the framework to support \textbf{large-group eye-tracking in classroom environments}, implementing both post-session and real-time analytics for instructors to gain insights into their students and teaching.\newline

The system consists of three primary components:

\begin{itemize}
    \item \textbf{Instructor Dashboard with RBAC:} Presents gaze and engagement data in a clear, interpretable format.
    \item \textbf{Post-Session Analytics Module:} Provides aggregated insights to evaluate attention patterns and group interactions after class.
    \item \textbf{Real-Time Analytics Module:} Delivers live feedback to instructors during lectures to support adaptive teaching.
\end{itemize}

Together, these components aim to enhance instructors’ understanding of student engagement and contribute to research on attention and collaboration in learning environments.

\subsection{Objectives}
This project aims to extend the existing SocialEyes system to support real-time session analytics through the integration of a lightweight computer vision algorithm that can efficiently map gaze points from egocentric to central views without compromising accuracy. Verification tests will focus on ensuring that the implemented models will perform correctly and consistently under real-time conditions.

Secondary objectives include the creation of the instructor dashboard for visualizing gaze data and providing analytics. Together, these components aim for easy interpretation of gaze-based feedback for instructors to provide useful feedback.

Certain objectives will intentionally be excluded due to project constraints. For instance, full verification of the underlying SuperPoint and SuperGlue models and other third-party libraries will not be performed, as these components are assumed to have been previously validated by their original developers. Additionally, large-scale usability testing with diverse participant groups is beyond the project’s current timeframe and resources. 



\subsection{Extras}
This project will include the following extra deliverables that were confirmed during the initial meeting with the project supervisors.

\begin{enumerate}
    \item \textbf{Code Walkthrough Report} \\
    This will be a thorough documentation of the files and folders in the GitHub repository associated with the project.
    
    \item \textbf{User Instruction Video} \\
    This will be an instructional video demonstrating how to interact with the dashboard deliverable of the project for each of the user roles (e.g., instructor).
\end{enumerate}


\subsection{Relevant Documentation}
The Verification and Validation (V\&V) Plan is dependent on and supported by several key project documents that define the foundation, scope, and success criteria for the group eye-tracking learning platform. These documents collectively guide how requirements are verified and validated throughout the development process.
\newline

\textbf{Software Requirements Specification (SRS)}
\newline
\newline
The \textit{Software Requirements Specification (SRS)} defines the functional and non-functional requirements of the system, serving as the primary reference for the verification process. It specifies what the system must do, how it should perform, and the constraints under which it must operate. Each verification test will be mapped to one or more requirements defined in the SRS.
\newline

\textbf{Development Plan}
\newline
\newline
The \textit{Development Plan} outlines the project’s implementation roadmap, including project decomposition, proof-of-concept demonstrations, and the technologies used to conduct it. It informs the scheduling and prioritization of verification and validation activities, ensuring that testing aligns with the overall development timeline and that each module is verified at appropriate stages in the lifecycle.
\newline

\textbf{Problem Statement and Goals Document}
\newline
\newline
The \textit{Problem Statement and Goals Document} defines the motivation, context, and intended impact of the project. It provides the conceptual foundation for validation activities, ensuring that the developed system not only meets its technical specifications but also fulfills its educational and research goals. This document reinforces the relevance of the V\&V Plan by providing a basis for why the tests being verified are crucial for achieving the project’s goals.
\newline

\section{Plan}

\wss{Introduce this section.  You can provide a roadmap of the sections to
  come.}

\subsection{Verification and Validation Team}

The Verification and Validation process for this project will involve both the planning and testing teams, as well as the project supervisor and members of the research group.\\

\textbf{Planning team:} Stanley Chen, Ann Shi, and Ibrahim Sahi will be responsible for developing review plans for SRS, VnV, and other documentation, as well as plans for design, implementation and software verification.\\

\textbf{Testing team:} Manan Sharma and Angela Zang will focus on developing and executing system tests that confirm the implementation satisfies the functional and non-functional requirements defined in the SRS.\\

The supervisor and research collaborators will provide support through reviewing and validating the planned systems.

\subsection{SRS Verification}

To ensure that the Software Requirements Specification is accurate, comprehensive, and aligned with the project’s goals, a two-stage review process will be used: a preliminary peer review followed by a structured team review meeting.\\

\textbf{Peer Review}\\

An initial peer review will be conducted by classmates to identify issues in readability, organization, and completeness. This step helps ensure that the SRS communicates effectively to a general audience and that the flow, terminology, and formatting are clear.\\

\textbf{Research Team Review Meeting}\\

A review meeting will be held with the research team during the routine meeting time. This session will combine the technical, instructional, and managerial perspectives required to validate the SRS in a single coordinated effort. During the meeting, a condensed version of the SRS will be presented, highlighting key sections such as system scope, functional and non-functional requirements, datasets, technology dependencies and goals. The review will be guided using a task-based inspection, structured around each participant’s perspective.\\

Instructors will verify that the system goals, use cases, and requirements align well with their needs as the primary users. Researchers will assess the logical soundness of the proposed workflows and ensure that data collection and analysis described in the SRS are technically consistent. The Supervisor will evaluate the overall feasibility of the system, ensuring that the listed technologies, environments, and resources are accessible and that the project remains within defined scope and constraints.\\

Feedback will be collected in real time and logged into the project’s issue tracker for traceability.\\

\subsection{Design Verification}

\wss{Plans for design verification}

\wss{The review will include reviews by your classmates}

\wss{Create a checklists?}

\subsection{Verification and Validation Plan Verification}

\wss{The verification and validation plan is an artifact that should also be
verified.  Techniques for this include review and mutation testing.}

\wss{The review will include reviews by your classmates}

\wss{Create a checklists?}

\subsection{Implementation Verification}

\wss{You should at least point to the tests listed in this document and the unit
  testing plan.}

\wss{In this section you would also give any details of any plans for static
  verification of the implementation.  Potential techniques include code
  walkthroughs, code inspection, static analyzers, etc.}

\wss{The final class presentation in CAS 741 could be used as a code
walkthrough.  There is also a possibility of using the final presentation (in
CAS741) for a partial usability survey.}

\subsection{Automated Testing and Verification Tools}

Automation is a central part of the verification strategy, ensuring that every commit is tested consistently. The following tools and frameworks will be used throughout development.

\begin{table}[htbp]
\centering
\caption{Automated Testing and Verification Tools}
\begin{tabularx}{\textwidth}{|p{3cm}|p{4cm}|X|}
\hline
\textbf{Tool} & \textbf{Purpose} & \textbf{Verification Activity} \\
\hline
GitHub Actions & Continuous integration pipeline that runs automatically on each pull request & Executes all test suites, linting, and build checks. Prevents merges if tests fail. \\
\hline
PyTest & Python testing framework for backend and data processing modules & Unit tests for data parsing, computation accuracy, and real-time gaze mapping \\
\hline
Jest and React Testing Library & Front-end testing for user interface components and data visualizations & Verifies correct rendering and state management on the instructor dashboard \\
\hline
ESLint, Prettier, and flake8 & Static code analysis and style enforcement tools & Ensures consistent syntax, formatting, and code quality \\
\hline
Valgrind or cProfile & Profiling and memory analysis tools & Identifies performance bottlenecks and inefficiencies in real-time data handling \\
\hline
Coverage.py and Codecov & Code coverage tracking utilities & Measures test completeness and identifies untested portions of the codebase \\
\hline
Docker & Reproducible environment for testing and deployment & Verifies installability and portability of the full system across different machines \\
\hline
\end{tabularx}
\end{table}

At the end of each development iteration, the CI dashboard will export a summary of test coverage, lint violations, and failed cases. These results will be included in the final VnV Report as objective evidence of verification and overall system reliability.

\subsection{Software Validation}

The validation of the enhanced SocialEyes framework will primarily be conducted through regular review sessions and iterative demonstrations with project supervisors Dr. Lauren Fink and Dr. Irene Yuan. Weekly meetings will serve as checkpoints to assess progress against the requirements outlined in the Software Requirements Specification (SRS), ensuring that the evolving system aligns with both the technical and research goals of the project.
\newline

A formal Rev 0 demonstration will be presented to the supervisors once completed. This demo will be used to validate that the core system features—real-time gaze analytics, post-session data visualization, and the instructor dashboard—accurately reflect the intended requirements and provide value within a classroom context. Feedback from this session will guide subsequent iterations and refinement of the system.
In addition to supervisor reviews, user validation will be obtained through feedback from a postdoctoral researcher affiliated with the SocialEyes project who has experience conducting lectures and an interest in applying this technology to teaching contexts. This will help validate the system’s usability and relevance from an instructor’s perspective.
\newline


\section{System Tests}

\wss{There should be text between all headings, even if it is just a roadmap of
the contents of the subsections.}

\subsection{Tests for Functional Requirements}

\wss{Subsets of the tests may be in related, so this section is divided into
  different areas.  If there are no identifiable subsets for the tests, this
  level of document structure can be removed.}

\wss{Include a blurb here to explain why the subsections below
  cover the requirements.  References to the SRS would be good here.}

\subsubsection{Area of Testing1}

\wss{It would be nice to have a blurb here to explain why the subsections below
  cover the requirements.  References to the SRS would be good here.  If a section
  covers tests for input constraints, you should reference the data constraints
  table in the SRS.}
		
\paragraph{Title for Test}

\begin{enumerate}

\item{test-id1\\}

Control: Manual versus Automatic
					
Initial State: 
					
Input: 
					
Output: \wss{The expected result for the given inputs.  Output is not how you
are going to return the results of the test.  The output is the expected
result.}

Test Case Derivation: \wss{Justify the expected value given in the Output field}
					
How test will be performed: 
					
\item{test-id2\\}

Control: Manual versus Automatic
					
Initial State: 
					
Input: 
					
Output: \wss{The expected result for the given inputs}

Test Case Derivation: \wss{Justify the expected value given in the Output field}

How test will be performed: 

\end{enumerate}

\subsubsection{Area of Testing2}

...

\subsection{Tests for Nonfunctional Requirements}

\wss{The nonfunctional requirements for accuracy will likely just reference the
  appropriate functional tests from above.  The test cases should mention
  reporting the relative error for these tests.  Not all projects will
  necessarily have nonfunctional requirements related to accuracy.}

\wss{For some nonfunctional tests, you won't be setting a target threshold for
passing the test, but rather describing the experiment you will do to measure
the quality for different inputs.  For instance, you could measure speed versus
the problem size.  The output of the test isn't pass/fail, but rather a summary
table or graph.}

\wss{Tests related to usability could include conducting a usability test and
  survey.  The survey will be in the Appendix.}

\wss{Static tests, review, inspections, and walkthroughs, will not follow the
format for the tests given below.}

\wss{If you introduce static tests in your plan, you need to provide details.
How will they be done?  In cases like code (or document) walkthroughs, who will
be involved? Be specific.}

\subsubsection{Area of Testing1}
		
\paragraph{Title for Test}

\begin{enumerate}

\item{test-id1\\}

Type: Functional, Dynamic, Manual, Static etc.
					
Initial State: 
					
Input/Condition: 
					
Output/Result: 
					
How test will be performed: 
					
\item{test-id2\\}

Type: Functional, Dynamic, Manual, Static etc.
					
Initial State: 
					
Input: 
					
Output: 
					
How test will be performed: 

\end{enumerate}

\subsubsection{Area of Testing2}

...

\subsection{Traceability Between Test Cases and Requirements}

\wss{Provide a table that shows which test cases are supporting which
  requirements.}

\section{Unit Test Description}

\wss{This section should not be filled in until after the MIS (detailed design
  document) has been completed.}

\wss{Reference your MIS (detailed design document) and explain your overall
philosophy for test case selection.}  

\wss{To save space and time, it may be an option to provide less detail in this section.  
For the unit tests you can potentially layout your testing strategy here.  That is, you 
can explain how tests will be selected for each module.  For instance, your test building 
approach could be test cases for each access program, including one test for normal behaviour 
and as many tests as needed for edge cases.  Rather than create the details of the input 
and output here, you could point to the unit testing code.  For this to work, you code 
needs to be well-documented, with meaningful names for all of the tests.}

\subsection{Unit Testing Scope}

\wss{What modules are outside of the scope.  If there are modules that are
  developed by someone else, then you would say here if you aren't planning on
  verifying them.  There may also be modules that are part of your software, but
  have a lower priority for verification than others.  If this is the case,
  explain your rationale for the ranking of module importance.}

\subsection{Tests for Functional Requirements}

\wss{Most of the verification will be through automated unit testing.  If
  appropriate specific modules can be verified by a non-testing based
  technique.  That can also be documented in this section.}

\subsubsection{Module 1}

\wss{Include a blurb here to explain why the subsections below cover the module.
  References to the MIS would be good.  You will want tests from a black box
  perspective and from a white box perspective.  Explain to the reader how the
  tests were selected.}

\begin{enumerate}

\item{test-id1\\}

Type: \wss{Functional, Dynamic, Manual, Automatic, Static etc. Most will
  be automatic}
					
Initial State: 
					
Input: 
					
Output: \wss{The expected result for the given inputs}

Test Case Derivation: \wss{Justify the expected value given in the Output field}

How test will be performed: 
					
\item{test-id2\\}

Type: \wss{Functional, Dynamic, Manual, Automatic, Static etc. Most will
  be automatic}
					
Initial State: 
					
Input: 
					
Output: \wss{The expected result for the given inputs}

Test Case Derivation: \wss{Justify the expected value given in the Output field}

How test will be performed: 

\item{...\\}
    
\end{enumerate}

\subsubsection{Module 2}

...

\subsection{Tests for Nonfunctional Requirements}

\wss{If there is a module that needs to be independently assessed for
  performance, those test cases can go here.  In some projects, planning for
  nonfunctional tests of units will not be that relevant.}

\wss{These tests may involve collecting performance data from previously
  mentioned functional tests.}

\subsubsection{Module ?}
		
\begin{enumerate}

\item{test-id1\\}

Type: \wss{Functional, Dynamic, Manual, Automatic, Static etc. Most will
  be automatic}
					
Initial State: 
					
Input/Condition: 
					
Output/Result: 
					
How test will be performed: 
					
\item{test-id2\\}

Type: Functional, Dynamic, Manual, Static etc.
					
Initial State: 
					
Input: 
					
Output: 
					
How test will be performed: 

\end{enumerate}

\subsubsection{Module ?}

...

\subsection{Traceability Between Test Cases and Modules}

\wss{Provide evidence that all of the modules have been considered.}
				
\bibliographystyle{plainnat}

\bibliography{../../refs/References}

\newpage

\section{Appendix}

This is where you can place additional information.

\subsection{Symbolic Parameters}

The definition of the test cases will call for SYMBOLIC\_CONSTANTS.
Their values are defined in this section for easy maintenance.

\subsection{Usability Survey Questions?}

\wss{This is a section that would be appropriate for some projects.}

\newpage{}
\section*{Appendix --- Reflection}

\wss{This section is not required for CAS 741}

The information in this section will be used to evaluate the team members on the
graduate attribute of Lifelong Learning.

The purpose of reflection questions is to give you a chance to assess your own
learning and that of your group as a whole, and to find ways to improve in the
future. Reflection is an important part of the learning process.  Reflection is
also an essential component of a successful software development process.  

Reflections are most interesting and useful when they're honest, even if the
stories they tell are imperfect. You will be marked based on your depth of
thought and analysis, and not based on the content of the reflections
themselves. Thus, for full marks we encourage you to answer openly and honestly
and to avoid simply writing ``what you think the evaluator wants to hear.''

Please answer the following questions.  Some questions can be answered on the
team level, but where appropriate, each team member should write their own
response:


\begin{enumerate}
  \item What went well while writing this deliverable? 
  \item What pain points did you experience during this deliverable, and how
    did you resolve them?
  \item What knowledge and skills will the team collectively need to acquire to
  successfully complete the verification and validation of your project?
  Examples of possible knowledge and skills include dynamic testing knowledge,
  static testing knowledge, specific tool usage, Valgrind etc.  You should look to
  identify at least one item for each team member.
  \item For each of the knowledge areas and skills identified in the previous
  question, what are at least two approaches to acquiring the knowledge or
  mastering the skill?  Of the identified approaches, which will each team
  member pursue, and why did they make this choice?
\end{enumerate}

\end{document}
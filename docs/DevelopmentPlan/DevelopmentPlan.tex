\documentclass{article}

\usepackage{booktabs}
\usepackage{tabularx}

\title{Development Plan\\\progname}

\author{\authname}

\date{}

%% Comments

\usepackage{color}

\newif\ifcomments\commentstrue %displays comments
%\newif\ifcomments\commentsfalse %so that comments do not display

\ifcomments
\newcommand{\authornote}[3]{\textcolor{#1}{[#3 ---#2]}}
\newcommand{\todo}[1]{\textcolor{red}{[TODO: #1]}}
\else
\newcommand{\authornote}[3]{}
\newcommand{\todo}[1]{}
\fi

\newcommand{\wss}[1]{\authornote{magenta}{SS}{#1}} 
\newcommand{\plt}[1]{\authornote{cyan}{TPLT}{#1}} %For explanation of the template
\newcommand{\an}[1]{\authornote{cyan}{Author}{#1}}

%% Common Parts

\newcommand{\progname}{Software Engineering} % PUT YOUR PROGRAM NAME HERE
\newcommand{\authname}{Team 21, Visionaries
\\ Ahmed Sahi
\\ Angela Zeng
\\ Ann Shi
\\ Manan Sharma
\\ Stanley Chen} % AUTHOR NAMES                  

\usepackage{hyperref}
    \hypersetup{colorlinks=true, linkcolor=blue, citecolor=blue, filecolor=blue,
                urlcolor=blue, unicode=false}
    \urlstyle{same}
                                


\begin{document}

\maketitle

\begin{table}[hp]
\caption{Revision History} \label{TblRevisionHistory}
\begin{tabularx}{\textwidth}{llX}
\toprule
\textbf{Date} & \textbf{Developer(s)} & \textbf{Change}\\
\midrule
09/23/2025 & All & Add first draft of development plan\\
... & ... & ...\\
\bottomrule
\end{tabularx}
\end{table}

\newpage{}

\wss{Put your introductory blurb here.  Often the blurb is a brief roadmap of
what is contained in the report.}

\wss{Additional information on the development plan can be found in the
\href{https://gitlab.cas.mcmaster.ca/courses/capstone/-/blob/main/Lectures/L02b_POCAndDevPlan/POCAndDevPlan.pdf?ref_type=heads}
{lecture slides}.}

\section{Confidential Information?}

No confidential information to protect.

\section{IP to Protect}

\hspace{\parindent}The software and documentation files for this project are protected by copyright (accessed at \href{https://github.com/mansha71/CapstoneProject/blob/main/LICENSE}{ link}). The License denies all parties the rights to modify, merge, publish, distribute, sub license, or sell the software without explicit permission.

\section{Copyright License}

See 'LICENSE' file found
\href{https://github.com/mansha71/CapstoneProject/blob/main/LICENSE}{here}.

\section{Team Meeting Plan}

\hspace{\parindent} The team plans to meet twice a week: once during the designated tutorial time (Thursday 4:30 - 6:30 PM) and once on weekends (Saturday 12:00 PM, subject to change weekly). These meetings will generally be held on-line, unless there is a specific need for an in-person meeting. The team will hold meetings with the project supervisors roughly twice a week initially, projected to change to weekly as more technical help is required at an advanced stage in the project. These meetings will be held online when discussing logistical concerns and in person when working with the physical equipment.

\section{Team Communication Plan}

\begin{itemize}

    \item \textbf{Issues}: GitHub 
    \item \textbf{Team Meetings/Discussions}: Discord
    \item \textbf{Meetings with Supervisors}: MS Teams
    
\end{itemize}

\section{Team Member Roles}

As a team, we have highlighted the following specific roles and responsibilities which will be designated to members based on their individual strengths:

\begin{itemize}
    \item \textbf{Liaison} \\
    Serves as the primary communicator between the project team and the supervisors. Responsible for providing updates, clarifying requirements, and addressing concerns in a timely manner.
    
    \item \textbf{UI/UX Lead} \\
    Oversees the dashboard portion of the project, ensuring the user interface is usable, accessible, and visually consistent. Works closely with the backend lead to integrate functionality with design.

    \item \textbf{Project Manager} \\
    Coordinates workflows, arranges meetings, develops agendas, assigns tasks, and tracks milestones. Ensures the project remains on schedule and that team members stay aligned.

    \item \textbf{Study Development Lead} \\
    Designs the structure of the research study and guides how collected data will be analyzed. Collaborates with the team to translate raw data into meaningful insights.

    \item \textbf{Backend Lead} \\
    Manages the backend architecture and real-time processing components of the project. Oversees backend development while coordinating with the UI/UX lead to ensure seamless integration.
\end{itemize}

\section{Workflow Plan}
\subsection*{Version Control}
We will use Git with a protected \texttt{main} branch. Our work will happen on \texttt{feature/*} and \texttt{bug/*} branches. All pull requests must receive one approval before they can be merged. An issue will be resolved once the PR is approved, the CI passes, the branch is up to date, and the linked issue is closed. Branch naming will follow the format:
\texttt{feature/<scope>-<short-title>} and \texttt{bug/<scope>-<short-title>}.

\subsection*{Issue Tracking and Project Management}
We will use GitHub Issues and a Kanban board with the columns Backlog, In progress, In review, and Done. Each issue will follow a template with fields: summary, context, acceptance criteria. Labels will be standardized: type:feature, type:bug, type:docs, prio:high, prio:med, prio:low, status:blocker. Commit messages will reference the issue number: \texttt{[\#ISSUE] short imperative description}. Example: [\#45-fixing-login-button]. Tags will be created for each stable milestone and release to mark progress and aid reproducibility.

\subsection*{Reviews and Quality Bar}
Every PR will link to it's relating issue, pass CI, and receive one approval. Reviewers will check acceptance criteria and code style. Linting and formatting will be enforced with ESLint and Prettier. Small fixes will be pushed to the PR branch, while larger changes will open a follow-up issue.

\subsection*{Continuous Integration}
We will run GitHub Actions on every PR. The jobs will include \texttt{lint}, \texttt{test}, and \texttt{build}. A PR will not merge unless all jobs succeed. Secrets and environment variables will be stored in GitHub Actions Secrets to keep them secure. Environments will require approval for deploy jobs if added later.

\subsection*{Releases}
We will follow continuous delivery without fixed sprints and when a milestone finishes we will tag a release (\texttt{v0.1.0}, \texttt{v0.2.0}, etc.). Release notes will be generated from merged PRs and any hotfixes will use \texttt{bug/*} branches and patch tags.

\section{Project Decomposition and Scheduling}

\subsection*{Project Decomposition}
We will decompose the project into major functional features so work can be distributed and tracked clearly:
\begin{itemize}
  \item \textbf{User Interface:} building the frontend views, navigation, and forms.
  \item \textbf{Data Processing:} implementing the logic to handle and analyze input data.
  \item \textbf{Backend Services:} creating APIs, database models, and integration logic.
  \item \textbf{Testing:} writing unit, integration, and end-to-end tests to ensure quality.
  \item \textbf{Documentation:} maintaining project documentation and developer guides.
\end{itemize}

\subsection*{Scheduling}
We will use GitHub Projects to manage our schedule and track progress through a
Kanban board: \url{https://github.com/users/mansha71/projects/5/views/1}.  
The schedule will align with both course deadlines and internal milestones:
\begin{itemize}
  \item \textbf{Early Phase:} Project setup and initial development environment configuration.
  \item \textbf{Middle Phase:} Core feature implementation across backend, frontend, and data processing.
  \item \textbf{Late Phase:} Integration, testing, and refinement of features.
  \item \textbf{Final Phase:} Documentation, polishing, and preparation for the proof of concept demo.
\end{itemize}

\section{Proof of Concept Demonstration Plan}

For context, our POC will consist of roughly the following steps:
\begin{enumerate}
  \item Connect a single pair of eye-tracking goggles and stream data into the system.
  \item Display gaze data on an instructor dashboard with basic real-time visualizations.
  \item Log session data for later post-session analysis.
  \item Provide a live demo where data from one user updates the dashboard view in real time.
\end{enumerate}

\noindent
The following are the primary risks and how potential results from the POC could mitigate them:
\begin{enumerate}
  \item \textbf{Real-time analytics may be too computationally complex.} If the pipeline cannot keep up with live data, we will reduce the frequency of incoming data, apply lightweight filtering algorithms, or pre-process the gaze data before visualization. This will help us identify the level of complexity the system can realistically handle and guide future design choices.
  \item \textbf{Eye-tracking hardware may malfunction during the demo.} Hardware errors or connection issues could prevent live data capture. To mitigate this, we will prepare recorded gaze data as a backup input source, ensuring the dashboard and analytics pipeline can still be demonstrated even without the live feed.
  \item \textbf{Privacy concerns may arise with identifiable gaze data.} Since gaze data can be linked to individuals, we will anonymize all outputs in the POC by removing names, IDs, or direct mappings to student information. This will allow us to demonstrate functionality while showing awareness of privacy and ethical requirements.
\end{enumerate}

Other smaller risks include:
\begin{itemize}
  \item \textbf{Integration challenges with multiple devices later.} Scaling from one device to many may create synchronization and performance issues. To mitigate this, we will first validate the pipeline with a single device, then gradually extend the architecture to support multiple inputs.
  \item \textbf{Generalizability to different lecture hall setups.} Different classrooms may use multiple screens or have varied seating arrangements. We will design dashboard components to be modular and adaptable so layouts can be reconfigured without requiring major code changes.
\end{itemize}

\section{Expected Technology}

We will use a mix of programming languages, frameworks, and tools to implement our system:

\begin{itemize}
  \item \textbf{Frontend:} The instructor dashboard will be built with React and
  TypeScript, providing a responsive interface and real-time visualizations.
  \item \textbf{Backend:} A Python backend (using FastAPI or Flask) will handle
  data ingestion from the eye-tracker, manage sessions, and serve analytics to
  the dashboard.
  \item \textbf{Data Processing:} We will acquire the data from the eye-tracking glasses, and create a processing pipeline using Python libraries such as NumPy and Pandas for data manipulation and analysis.
  \item \textbf{Data Storage and Streaming:} A lightweight relational database (SQLite or PostgreSQL) will be used to log session data for post-session analytics. For scalability and handling real-time streams, we will explore Apache Kafka if it fits the course infrastructure.
  \item \textbf{Visualization:} Simple data visualizations will be implemented first using React charting libraries such as Chart.js or Recharts. More advanced options (e.g., D3.js) may be added if required.
  \item \textbf{Collaboration and CI/CD:} GitHub will be used for version control, issue tracking, and project management. GitHub Actions will provide continuous integration with automated linting and testing to maintain code quality.
  \item \textbf{Deployment:} The system will initially run locally for development and POC purposes. Docker may be introduced later to improve reproducibility and deployment flexibility.
\end{itemize}

Note: Additional technologies may be added as the project evolves.
\section{Coding Standard}

We will follow consistent coding practices to keep the project maintainable and consistent. Style guides, linting, and comments will be used to ensure clarity throughout the codebase. Commit messages will follow a standard format, and pull requests will be reviewed before merging.

\newpage{}

\section*{Appendix --- Reflection}


The purpose of reflection questions is to give you a chance to assess your own
learning and that of your group as a whole, and to find ways to improve in the
future. Reflection is an important part of the learning process.  Reflection is
also an essential component of a successful software development process.  

Reflections are most interesting and useful when they're honest, even if the
stories they tell are imperfect. You will be marked based on your depth of
thought and analysis, and not based on the content of the reflections
themselves. Thus, for full marks we encourage you to answer openly and honestly
and to avoid simply writing ``what you think the evaluator wants to hear.''

Please answer the following questions.  Some questions can be answered on the
team level, but where appropriate, each team member should write their own
response:


\begin{enumerate}
    \item Why is it important to create a development plan prior to starting the
    project?

\textbf{Stanley:} It is important to create a development plan because a development plan reduces chaos and helps the team organize themselves better. It allows for smoother integration of parts that have been planned, and the team also has a direction and set of expectations thta they can rely on. 

\textbf{Manan:} A development plan is important because it helps the team organize ideas and set clear expectations of how work will be completed. Following a plan reduces conflict later on and helps keep the team on track of the goal. Creating this before the project starts allows the team to have a clear direction and avoid loss in efficiency.

\textbf{Angela:} Creating a development plan is important because it gives the team structure and direction before the project begins. It helps make sure that everyone is on the same page on expectations and gives a clear reference point as the project progresses.

    \item In your opinion, what are the advantages and disadvantages of using
    CI/CD?

\textbf{Stanley:} An advantage of CI/CD would be that bugs are caught faster because of the automated testing. Because of this, code quality is generally better. Pipelines are also a lot more reliable than manual testing. A disadvantage of CI/CD would be cost. Setting it up takes time, and it running tests every commit could be costly.

\textbf{Manan:} I think using CI/CD makes it easier to reduce technical debt and catch issues early. It also helps enfornce consistency in code quality. However, setting up CI/CD can be time-consuming and may require additional resources, which could be a disadvantage for smaller teams or projects with limited budgets.

\textbf{Angela:} CI/CD is an advantage because it helps maintain the same code quality and lessens the chances of errors slipping through. It also helps with catching bugs in the beginning. The disadvantage is that setting it up requires time and effort which can be frustrating especially for minor changes.

    \item What disagreements did your group have in this deliverable, if any,
    and how did you resolve them?

\textbf{Stanley:} During this deliverable, there were not any major disagreements, as we were able to allocate work smoothly since there was not too much work to do yet.

\textbf{Manan:} Our team was able to avoid major disagreements during this deliverable as we all hold similar views on how one should approach a project like this. We were able to divide the work equally and discuss about our development plan together.

\textbf{Angela:} Our team didn’t have major disagreements for the development plan. We were able to share ideas openly and divide up the work in a way that felt fair, which helped things go smoothly.  

\end{enumerate}

\newpage{}

\section*{Appendix --- Team Charter}


\subsection*{External Goals}

Our goal for this course is to achieve a 12 on the McMaster GPA scale. We also hope to create a project that would be impressive on our resumes and be something that could be talked about during our interviews for new grad jobs. 

\subsection*{Attendance}

\subsubsection*{Expectations}

Our team plans to meet once a week for roughly two hours every Thursday at 4:30 during our tutorial time. We also scheduled a weekly meeting every Saturday for an hour so that we could discuss all the work that we’ve been working on individually and to finalize certain changes. In terms of punctuality, we’ve given a grace period of 10 minutes for being late in terms of being late, since there may be technical difficulties when joining or being pardoned for a bathroom break. 

\subsubsection*{Acceptable Excuse}

Acceptable excuses would include any medical, personal, or family related issues that arise. It would be like a Self-Report MSAF. That or, if a team member lets our group know in advance, they can opt out of a meeting for any reasonable issues.

\subsubsection*{In Case of Emergency}

If a team member cannot meet up, they would be informed by one of the other team members about what had been discussed in the meeting and informed about what work they had been assigned. They would still need to complete the assigned work by the deadline assigned. If they cannot do that because of emergency reasons, they will be assigned more work in latter stages of the project to make up for the missed effort. 

\subsection*{Accountability and Teamwork}

\subsubsection*{Quality} 

Team members are expected to come prepared with all materials and expectations set from the previous meetings. The quality of the work should at least satisfy all the requirements set in the rubric, as well as any expectations set by the team in previous meetings. 

\subsubsection*{Attitude}

Team members are expected to treat each other with respect and handle conflict maturely. All members will be respected and have a voice, and all opinions will be considered. They should be able to work co-operatively when assigned tasks together, as well as be punctual and respect each other’s time. 

\subsubsection*{Stay on Track}

Team members will be assigned tasks so that they can contribute around an equal amount. To ensure quality work, we will all go over the individual work that we contributed together as a group so that any poorly done will not be overlooked. There will not be a reward and punishment system for the quality of work. Members are expected to perform, and if they cannot, they can rely on other group members for assistance. As long as all of the work assigned has been completed, then there will be no issues. If a group member is not co-operating, it may be addressed to the group first, and if need be, could be bought up to administrative figures such as the TA’s. 

\subsubsection*{Team Building}

Most of the members worked together at Citi as interns during the summer, so they have experience working together. As per team building, we have already hiked together and plan on meeting up more during the school year to play sports such as badminton. 

\subsubsection*{Decision Making} 

How our group handles decision making is through discussion during the meetings and voting. When we were choosing our project idea, we decided to use a decision matrix to see the most desired project idea. 

\end{document}
